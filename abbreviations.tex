%!TEX TS-program = xelatex
%!TEX encoding = UTF-8 Unicode

\usepackage{xspace} \xspaceaddexceptions{”}
\usepackage{ifthen}

\newcommand{\ch}[1]		{chapter~\ref{#1}}
\newcommand{\sect}[1]		{section~\ref{#1}}
\newcommand{\fig}[1]		{figure~\vref{#1}}
\newcommand{\Fig}[1]		{Figure~\vref{#1}}
\newcommand{\tbl}[1]		{table~\vref{#1}}

\newcommand{\ie}			{i.e. }
\newcommand{\eg}			{e.g. }

\newcommand{\qq}			{\qquad}

\newcommand{\spitz}[1]		{\ensuremath{\langle}#1\ensuremath{\rangle}}

\newcommand{\mehrzeilen}[1][1]{\enlargethispage{#1\baselineskip}}

% https://tex.stackexchange.com/questions/4386/defining-starred-versions-of-commands-macro
\usepackage{suffix}

% Zeilenabstand

\usepackage{setspace}
\newenvironment{enum}{\begin{enumerate} \singlespacing} {\end{enumerate}}
\newenvironment{items}{\begin{itemize} \singlespacing} {\end{itemize}}


% verbatim

\usepackage{verbatim}

\usepackage{alltt}
\newcommand{\klein}{\small}
\newenvironment{exakt}[1][\small]{\singlespacing#1\begin{alltt}}{\end{alltt}}

\usepackage{shortvrb}
\MakeShortVerb{\§}


%%%%%%%%%%%%%%%%%%%%%%%%%%%%%%%%

% xml commands
% use for any xml markup, brackets supplied
\newcommand{\xml}[1]{§<#1>§}
% use for milestone tags
\newcommand{\xms}[1]{§<#1/>§}
% closing element markup
\newcommand{\xmcl}[1]{§</#1>§}
% full xml markup example
\newcommand{\xmex}[1]{§#1§}
% opening and closing
\newcommand{\xmlpair}[1]{§<#1>§ and §</#1>§}
\WithSuffix\newcommand\xmlpair*[1]{§<#1>§~§</#1>§}
% xml entities
\newcommand{\xent}[1]{§\&#1;§}
% opening and closing with attribute
\newcommand{\xmlatt}[3]{§<#1>§ with the attribute §@#2="#3"§}
\newcommand{\xmsatt}[3]{§<#1/>§ with the attribute §@#2="#3"§}
\newcommand{\xmlpatt}[3]{§<#1>§ and §</#1>§ with the attribute §@#2="#3"§}
% attribute markup, the first argument is the element name
\newcommand{\attr}[2]{§@#2§}

% linewrap
\newcommand{\lwr}{\\\hspace{2mm}\unicode{↪}\hspace{2mm}}

% unicode points
\newcommand{\uc}[1]{U+#1}

\newcommand{\bold}{\textbf}
% ligature
\newcommand{\li}[1]{\bold{\{}#1\bold{\}}}

\newcommand{\xs}{\scriptsize}
\newcommand{\s}{\footnotesize}

%

\newcommand{\bs}{\textbackslash}
\newcommand{\tld}{\textasciitilde}

\newcommand{\tocspace}{\addtocontents{toc}{\protect\vspace{1mm}}}

\newcommand{\unicode}[1]{{\fontspec{Apple Symbols}{\Large #1}}}
\newcommand{\§}{{\char"00A7}}

%%%%%%%%%%%%%%%%

\newcommand{\htsc}[1]{\emph{#1}}
\newcommand{\lig}[1]{\fontspec{Hoefler Text}{\Large #1}}
\newcommand{\fraktur}[1]{{\fontspec{BreitkopfFraktur}{\LARGE #1}}}

%

\newenvironment{mainrule}{}{}
\newenvironment{mainruleLessImportant}{}{}
\newenvironment{clarification}{\s}{}
\newenvironment{exception}{\htsc{Exception:}}{}
\newenvironment{note}{\textbf{Please note:}}{}
\newenvironment{crossref}{\s\ensuremath{\longrightarrow}}{}

%

\newenvironment{sampleImage}[2][]{\parbox{\linewidth}{{\htsc{Example#1}} \\[3mm] \includegraphics[width=\linewidth]{#2}}}{}
\newenvironment{sampleImageSmall}[3][]{\parbox{\linewidth}{{\htsc{Example#1}} \\[3mm] \includegraphics[#2]{#3}}}{}

\newenvironment{example}[1][]{\htsc{Example#1} \\}{}
\newenvironment{exampleTest}[2][]{\parbox{\linewidth}{\htsc{Example #1} \\[3mm] #2}}{} % ??

\newenvironment{liste}[1][]{\htsc{List#1} \\}{}
\newenvironment{tabelle}[1][]{\htsc{Table#1} \\}{}

%
\usepackage{mdframed}
\definecolor{shadecolor}{gray}{0.95}
\mdfdefinestyle{mdfspe}{backgroundcolor=shadecolor, linecolor=shadecolor, innerleftmargin=4pt, innerrightmargin=4pt}

\newenvironment{typeLatin}{\begin{mdframed}[style=mdfspe]\begin{alltt}\s\begin{tabular}{@{}l}}{\end{tabular}\end{alltt}\end{mdframed}}

\newfontfamily{\greek}[Scale=0.95]{Courier New}
\newenvironment{typeGreek}{\begin{mdframed}[style=mdfspe]\begin{alltt}\greek\s\begin{tabular}{@{}l}}{\end{tabular}\end{alltt}\end{mdframed}}

\newenvironment{typeMath}{\begin{mdframed}[style=mdfspe]\begin{alltt}\begin{tabular}{l}}{\end{tabular}\end{alltt}\end{mdframed}}

%

\newfontfamily{\muh}[Scale=0.9]{DejaVu Serif}
\newcommand{\someText}{...} % {{\muh\textit{(some text)}}}
\newcommand{\untranscribedText}{...} % {{\muh\textit{(some untranscribed text)}}}
\newcommand{\notTranscribed}{{\muh\textit{(not transcribed)}}}
\newcommand{\missingText}[1]{{\muh\textit{(#1)}}}

%% Chinese bits
\newenvironment{typeChinese}{\begin{alltt}\s\begin{tabular}{@{}l}}{\end{tabular}\end{alltt}}

\newcommand{\chin}[1]{{\fontspec{Sun-ExtA}{#1}}}
\newcommand{\sunExtA}[1]{{\fontspec{Sun-ExtA}{#1}}}
\newcommand{\sunExtB}[1]{{\fontspec{Sun-ExtB}{#1}}}

\newcommand{\mincho}[1]{{\fontspec{MS Mincho}{#1}}}
\newcommand{\hira}[1]{{\fontspec{HiraMinPro-W3}{#1}}}

\newcommand{\f}[1]{\bold{#1}} % f für fett
\newcommand{\z}[1]{\chin{#1}} % z für Zeichen
