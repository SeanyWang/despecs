%!TEX TS-program = xelatex 
%!TEX encoding = UTF-8 Unicode 
%!TEX root = ../DESpecs.tex

\usepackage{xspace} \xspaceaddexceptions{”}
\usepackage{ifthen}

\newcommand{\ch}[1]		{chapter~\ref{#1}}
\newcommand{\sect}[1]		{section~\ref{#1}}
\newcommand{\fig}[1]		{figure~\vref{#1}}
\newcommand{\Fig}[1]		{Figure~\vref{#1}}
\newcommand{\tbl}[1]		{table~\vref{#1}}

\newcommand{\ie}			{i.e. }
\newcommand{\eg}			{e.g. }

\newcommand{\qq}			{\qquad}

\newcommand{\spitz}[1]		{\ensuremath{\langle}#1\ensuremath{\rangle}}

\newcommand{\mehrzeilen}[1][1]{\enlargethispage{#1\baselineskip}}


% Zeilenabstand

\usepackage{setspace} 
\newenvironment{enum}{\begin{enumerate} \singlespacing} {\end{enumerate}}
\newenvironment{items}{\begin{itemize} \singlespacing} {\end{itemize}}


% verbatim

\usepackage{verbatim}

\usepackage{alltt}
\newcommand{\klein}{\small}
\newenvironment{exakt}[1][\small]{\singlespacing#1\begin{alltt}}{\end{alltt}}

\usepackage{shortvrb}
\MakeShortVerb{\§}


%%%%%%%%%%%%%%%%%%%%%%%%%%%%%%%%

\newcommand{\bold}{\textbf}
\newcommand{\li}[1]{\bold{\{}#1\bold{\}}}

\newcommand{\xs}{\scriptsize}
\newcommand{\s}{\footnotesize}


\newcommand{\bs}{\textbackslash}
\newcommand{\tld}{\textasciitilde}

\newcommand{\tocspace}{\addtocontents{toc}{\protect\vspace{1mm}}}

\newcommand{\unicode}[1]{{\fontspec{Apple Symbols}{\Large #1}}}
\newcommand{\§}{{\char"00A7}}

%%%%%%%%%%%%%%%%

%\newfontfamily{\H}{Hoefler Text} 
\newcommand{\htsc}[1]{{\fontspec{Hoefler Text}{\large\scshape #1}}}
\newcommand{\lig}[1]{\fontspec{Hoefler Text}{\Large #1}}

\newenvironment{mainrule}{}{}
\newenvironment{mainruleLessImportant}{}{}
\newenvironment{clarification}{\s}{}
\newenvironment{exception}{\htsc{Exception:}}{}
\newenvironment{note}{\htsc{Please note:}}{}
\newenvironment{crossref}{\s\ensuremath{\longrightarrow}}{}


\newenvironment{sampleImage}[2][]{\parbox{\linewidth}{{\htsc{Example #1}} \\[3mm] \includegraphics[width=\linewidth]{#2}}}{}

\newenvironment{smallSampleImage}[2]{\parbox{\linewidth}{{\htsc{Example}} \\[3mm] \includegraphics[#1]{#2}}}{}

\newenvironment{example}[1][]{\htsc{Example #1} \\}{}
%\newenvironment{example}[1][]{\parbox{\linewidth}{{\fontspec{Hoefler Text}{\large\scshape Example #1}} \\[3mm] }}{}
\newenvironment{exampleTest}[2][]{\parbox{\linewidth}{\htsc{Example #1} \\[3mm] #2}}{}

\newenvironment{liste}[1][]{\htsc{List#1} \\}{}
\newenvironment{tabelle}[1][]{\htsc{Table#1} \\}{}

\newenvironment{typeLatin}{\begin{alltt}\s\begin{tabular}{@{}l}}{\end{tabular}\end{alltt}}

\newfontfamily{\greek}[Scale=0.95]{Courier New} 
\newenvironment{typeGreek}{\begin{alltt}\greek\s\begin{tabular}{@{}l}}{\end{tabular}\end{alltt}}

\newenvironment{typeMath}{\begin{alltt}\begin{tabular}{l}}{\end{tabular}\end{alltt}}


\newfontfamily{\muh}[Scale=0.9]{DejaVu Serif} 
\newcommand{\someText}{{\muh\textit{(some text)}}} % für Text, der im Buch, aber nicht im Bild ist
\newcommand{\untranscribedText}{{\muh\textit{(some untranscribed text)}}} % für Text, der im Bild, aber nicht in der Transkiption ist
\newcommand{\notTranscribed}{{\muh\textit{(not transcribed)}}} 
\newcommand{\missingText}[1]{{\muh\textit{(#1)}}} % andere Angaben
