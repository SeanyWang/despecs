%!TEX TS-program = xelatex 
%!TEX encoding = UTF-8 Unicode 

\documentclass[fontsize=11pt, paper=a4, 
DIV15,
normalheadings,
parskip=half-, 
pointlessnumbers]{scrartcl}

\usepackage[british]{babel} 

\usepackage{fontspec,xltxtra,xunicode} 
\defaultfontfeatures{Mapping=tex-text} 

\setromanfont[Mapping=tex-text]{DejaVu Serif}
\setsansfont[Scale=MatchLowercase,Mapping=tex-text]{Helvetica} 
\setmonofont[Scale=1.0]{Courier New} 

\frenchspacing

\usepackage{graphicx}
\graphicspath{{./Bilder/}}

\usepackage{longtable}

\usepackage{philokalia}

%%%

%!TEX TS-program = xelatex
%!TEX encoding = UTF-8 Unicode

\usepackage{xspace} \xspaceaddexceptions{”}
\usepackage{ifthen}

\newcommand{\ch}[1]		{chapter~\ref{#1}}
\newcommand{\sect}[1]		{section~\ref{#1}}
\newcommand{\fig}[1]		{figure~\vref{#1}}
\newcommand{\Fig}[1]		{Figure~\vref{#1}}
\newcommand{\tbl}[1]		{table~\vref{#1}}

\newcommand{\ie}			{i.e. }
\newcommand{\eg}			{e.g. }

\newcommand{\qq}			{\qquad}

\newcommand{\spitz}[1]		{\ensuremath{\langle}#1\ensuremath{\rangle}}

\newcommand{\mehrzeilen}[1][1]{\enlargethispage{#1\baselineskip}}


% Zeilenabstand

\usepackage{setspace}
\newenvironment{enum}{\begin{enumerate} \singlespacing} {\end{enumerate}}
\newenvironment{items}{\begin{itemize} \singlespacing} {\end{itemize}}


% verbatim

\usepackage{verbatim}

\usepackage{alltt}
\newcommand{\klein}{\small}
\newenvironment{exakt}[1][\small]{\singlespacing#1\begin{alltt}}{\end{alltt}}

\usepackage{shortvrb}
\MakeShortVerb{\§}


%%%%%%%%%%%%%%%%%%%%%%%%%%%%%%%%

% xml commands
% use for any xml markup, brackets supplied
\newcommand{\xml}[1]{§<#1>§}
% use for milestone tags
\newcommand{\xms}[1]{§<#1/>§}
% closing element markup
\newcommand{\xmcl}[1]{§</#1>§}
% full xml markup example
\newcommand{\xmex}[1]{§#1§}
% attribute markup, the first argument is the element name
\newcommand{\attr}[2]{§@#2§}

\newcommand{\bold}{\textbf}
% ligature
\newcommand{\li}[1]{\bold{\{}#1\bold{\}}}

\newcommand{\xs}{\scriptsize}
\newcommand{\s}{\footnotesize}

%

\newcommand{\bs}{\textbackslash}
\newcommand{\tld}{\textasciitilde}

\newcommand{\tocspace}{\addtocontents{toc}{\protect\vspace{1mm}}}

\newcommand{\unicode}[1]{{\fontspec{Apple Symbols}{\Large #1}}}
\newcommand{\§}{{\char"00A7}}

%%%%%%%%%%%%%%%%

\newcommand{\htsc}[1]{\emph{#1}}
\newcommand{\lig}[1]{\fontspec{Hoefler Text}{\Large #1}}
\newcommand{\fraktur}[1]{{\fontspec{BreitkopfFraktur}{\LARGE #1}}}

%

\newenvironment{mainrule}{}{}
\newenvironment{mainruleLessImportant}{}{}
\newenvironment{clarification}{\s}{}
\newenvironment{exception}{\htsc{Exception:}}{}
\newenvironment{note}{\textbf{Please note:}}{}
\newenvironment{crossref}{\s\ensuremath{\longrightarrow}}{}

%

\newenvironment{sampleImage}[2][]{\parbox{\linewidth}{{\htsc{Example#1}} \\[3mm] \includegraphics[width=\linewidth]{#2}}}{}
\newenvironment{sampleImageSmall}[3][]{\parbox{\linewidth}{{\htsc{Example#1}} \\[3mm] \includegraphics[#2]{#3}}}{}

\newenvironment{example}[1][]{\htsc{Example#1} \\}{}
\newenvironment{exampleTest}[2][]{\parbox{\linewidth}{\htsc{Example #1} \\[3mm] #2}}{} % ??

\newenvironment{liste}[1][]{\htsc{List#1} \\}{}
\newenvironment{tabelle}[1][]{\htsc{Table#1} \\}{}

%

\newenvironment{typeLatin}{\begin{alltt}\s\begin{tabular}{@{}l}}{\end{tabular}\end{alltt}}

\newfontfamily{\greek}[Scale=0.95]{Courier New}
\newenvironment{typeGreek}{\begin{alltt}\greek\s\begin{tabular}{@{}l}}{\end{tabular}\end{alltt}}

\newenvironment{typeMath}{\begin{alltt}\begin{tabular}{l}}{\end{tabular}\end{alltt}}

%

\newfontfamily{\muh}[Scale=0.9]{DejaVu Serif}
\newcommand{\someText}{...} % {{\muh\textit{(some text)}}}
\newcommand{\untranscribedText}{...} % {{\muh\textit{(some untranscribed text)}}}
\newcommand{\notTranscribed}{{\muh\textit{(not transcribed)}}}
\newcommand{\missingText}[1]{{\muh\textit{(#1)}}}

%% Chinese bits
\newenvironment{typeChinese}{\begin{alltt}\s\begin{tabular}{@{}l}}{\end{tabular}\end{alltt}}

\newcommand{\chin}[1]{{\fontspec{Sun-ExtA}{#1}}}
\newcommand{\sunExtA}[1]{{\fontspec{Sun-ExtA}{#1}}}
\newcommand{\sunExtB}[1]{{\fontspec{Sun-ExtB}{#1}}}

\newcommand{\mincho}[1]{{\fontspec{MS Mincho}{#1}}}
\newcommand{\hira}[1]{{\fontspec{HiraMinPro-W3}{#1}}}

\newcommand{\f}[1]{\bold{#1}} % f für fett
\newcommand{\z}[1]{\chin{#1}} % z für Zeichen


\begin{document}

\begin{center}
{\fontspec{Helvetica}{\LARGE \textbf{
Special Instructions for Heeschen 1990
\\[3mm]
(Addendum to Data Entry Specs 2.0) 
}}} \\[5mm]
\large Wolfgang Schmidle, Martin Thiering, Klaus Thoden, Malcolm D. Hyman

\normalsize Max Planck Institute for the History of Science, Berlin, Germany

\today
\end{center}

\section{Passages with Eipo Text}

\begin{mainrule}
The book contains many passages with line pairs, i.e. one line of Eipo text (normal size) and one line of translation (small text).  Mark the Eipo text lines with §<p> </p>§ and the translation text lines with §<p sm> </p>§.
\end{mainrule}

\begin{clarification}
Mark underlines with §<ul> </ul>§.
\end{clarification}

\vspace{5mm}
\begin{sampleImage}[: 0060.jpg]{eipo_60.jpg}

\vspace{-3mm}
\begin{typeLatin}
\xs\bold{<p>}Tenebik-ba, el \bold{<ul>}winebuka abukye\bold{</ul>} »na kil,« winebuka, ninye sik do kil, ... \bold{</p>} \\
\xs\bold{<p sm>}Sie=dachten-SW, er er=hatte=gemacht »meine Frau,« so=gemacht, Menschen ihr Ahn Frau, ... \bold{</p>} \\
\xs\bold{<mgl>}40\bold{</mgl>} \\
\xs\bold{<p>}asik balamnye dara, el asik ulamlirye, yukyuka yupe gekebokablirye, teleb ubnamle ... \bold{</p>} \\
\xs\bold{<p sm>}Weiler Gehen-ich TA, sie Weiler sie=ist=Ko, andere Sprache gehört=Ko, gut sie=w.=sein ... \bold{</p>} \\
\xs\bold{<p>}ninye naryuk kwebreibse-buk, na-di kwebreibse ate, naning-uk, na meme,« ... \bold{</p>} \\
\xs\bold{<p sm>}Menschen ich=S=nur ich=setzte-SW, ich-S ich=setzte weil, meins-nur, mein Tabu,« ... \bold{</p>} \\
\end{typeLatin}

%\xs <p>Tenekbik-ba, el <ul>winebuka abukye</ul> »na kil,« winebuka, ninye sik do kil, »na kil, na-de ninye</p> \\
%\xs <p sm>Sie=dachten-SW, er er=hatte=gemacht »meine Frau,« so=gemacht, Menschen ihr Ahn Frau, »meine Frau, ich-aber Menschen</p> \\
%\xs <mgl>40</mgl> \\
%\xs <p>asik balamnye dara, el asik ulamlirye, yukyuka yupe gekebokablirye, teleb ubnamle kwemdina ara,</p> \\
%\xs <p sm>Weiler Gehen-ich TA, sie Weiler sie=ist=Ko, andere Sprache gehört=Ko, gut sie=w.=sein Schöpfung T,</p> \\
%\xs <p>ninye naryuk kwebreibse-buk, na-di kwebreibse ate, naning-uk, na meme,« winyabikye dara, »na</p> \\
%\xs <p sm>Menschen ich=S=nur ich=setzte-SW, ich-S ich=setzte weil, meins-nur, mein Tabu,« sie=sagend TA, »ich</p> \\

\end{sampleImage}

\begin{note}
In these passages, do not insert a space after a period if there is none in the text, e.g. §sie=w.=sein§. Do not mark large spaces, i.e. type them as normal spaces.
\end{note}


\section{German Characters}

\begin{mainrule}
Please type the German characters ä, ö, ü and ß directly as Unicode characters.
\end{mainrule}

\vspace{3mm}
\begin{tabelle}[: \, German characters]

\vspace{-1mm}
%\begin{tabular}{@{}lcccccclc}
%character \hspace{6mm} & ä & ö & ü & Ä & Ö & Ü && ß \\[2mm]
%type as & §ä§ & §ö§ & §ü§ & §Ä§ & §Ö§ & §Ü§ && §ß§ \\[1mm]
%& \xs{U+00E4} & \xs{U+00F6} & \xs{U+00FC} & \xs{U+00E4} & \xs{U+00F6} & \xs{U+00FC} && \xs{U+00DF} \\ \\
%\end{tabular}
\begin{tabular}{@{}lccclc}
small letters \hspace{8mm} & ä & ö & ü && ß \\[2mm]
Unicode & \xs{U+00E4} & \xs{U+00F6} & \xs{U+00FC} && \xs{U+00DF} \\[4mm]
capital letters \hspace{8mm} & Ä & Ö & Ü \\[2mm]
Unicode & \xs{U+00C4} & \xs{U+00D6} & \xs{U+00DC} \\
\end{tabular}


\end{tabelle}

\end{document}
