%!TEX TS-program = xelatex 
%!TEX encoding = UTF-8 Unicode 

\documentclass[fontsize=11pt, paper=a4, 
  DIV15,
  normalheadings,
  parskip=half-, 
  pointlessnumbers]{scrartcl}

\usepackage[british]{babel} 

\usepackage{fontspec,xltxtra,xunicode} 
\defaultfontfeatures{Mapping=tex-text} 

\setromanfont[Mapping=tex-text]{DejaVu Serif}
\setsansfont[Scale=MatchLowercase,Mapping=tex-text]{Helvetica} 
\setmonofont[Scale=1.0]{Courier New} 

\frenchspacing

\usepackage{graphicx}
\graphicspath{{../../bilder/}}
\graphicspath{{./bilder/}}

\usepackage{longtable}

\usepackage{philokalia}

%%%

%!TEX TS-program = xelatex 
%!TEX encoding = UTF-8 Unicode 
%!TEX root = ../DESpecs.tex

\usepackage{xspace} \xspaceaddexceptions{”}
\usepackage{ifthen}

\newcommand{\ch}[1]		{chapter~\ref{#1}}
\newcommand{\sect}[1]		{section~\ref{#1}}
\newcommand{\fig}[1]		{figure~\vref{#1}}
\newcommand{\Fig}[1]		{Figure~\vref{#1}}
\newcommand{\tbl}[1]		{table~\vref{#1}}

\newcommand{\ie}			{i.e. }
\newcommand{\eg}			{e.g. }

\newcommand{\qq}			{\qquad}

\newcommand{\spitz}[1]		{\ensuremath{\langle}#1\ensuremath{\rangle}}

\newcommand{\mehrzeilen}[1][1]{\enlargethispage{#1\baselineskip}}


\newcommand{\unit}[1]		{\textsl{#1}}


% Zeilenabstand

\usepackage{setspace} 
\newenvironment{enum}{\begin{enumerate} \singlespacing} {\end{enumerate}}
\newenvironment{items}{\begin{itemize} \singlespacing} {\end{itemize}}


% verbatim

\usepackage{alltt}
\newcommand{\klein}{\small}
\newenvironment{exakt}[1][\small]{\singlespacing#1\begin{alltt}}{\end{alltt}}

\usepackage{shortvrb}
\MakeShortVerb{\§}


%%%%%%%%%%%%%%%%

\newenvironment{mainrule}{\textit{Rule}}{}

\newenvironment{example}{\textit{Example}}{}
\newenvironment{type}{\begin{alltt}}{\end{alltt}}

%\newfontfamily{\greek}[Scale=1]{Minion Pro} 
%\newfontfamily{\greek}[Scale=0.8]{Andale Mono} 
\newfontfamily{\greek}[Scale=0.8]{Courier New} 
\newenvironment{typeGreek}{\begin{alltt}\greek}{\end{alltt}}

\newenvironment{exception}{\textit{Exception}}{}

\newenvironment{clarification}{\textit{Clarification}}{}


\begin{document}

\begin{center}
  {\fontspec{Helvetica}{\LARGE \textbf{
        Special Instructions for "Ben Cao Jing Ji Zhu (\chin{本草經集注})”
        \\[3mm]
        (Addendum to Data Entry Specs for Chinese Text 2.1) 
  }}} \\[5mm]
  \large Klaus Thoden, Shih-Pei Chen, Georg Freise

  \normalsize Max Planck Institute for the History of Science, Berlin, Germany

  \today
\end{center}

\section{Footnote marks}
\begin{mainrule}
  This book contains footnotes with anchors in the main text. The
  footnote marks in the main text are surrounded by Chinese
  parentheses. Please tag them with §<a>§ and §</a>§.
\end{mainrule}

%% \begin{clarification}
%% Type the §<tf>§ and §</tf>§ tags on separate lines. On each page, type the first text flow before the second text flow.
%% \end{clarification}
\vspace{3mm}
%%\includegraphics[height=4cm]{image1}
\begin{sampleImageSmall}[\ 1: \, Footnote marks.]{height=10cm}{image1.png}
\end{sampleImageSmall}
%% \begin{tabular}{@{}ll}
%%   \parbox[b]{131mm}{
\begin{typeChinese}
  \f{<h3>}\z{玉屑}\f{</h3>}\\
  \f{<p>}\z{味甘,平,無毒。主除胃中熱、喘息、煩滿,止渴,屑如麻豆服之。久}\\
  \z{服輕身長年。生藍田,采無時。}\f{<sm>}\z{惡鹿角。}\f{</sm></p>}\\
  \f{<p><sm>}\z{此云玉屑,亦是以玉為屑,非應別一種物也。《仙經》服榖玉,有擣如米粒,乃以苦酒輩}\f{<a>}\z{﹝一﹞}\f{</a>,}\z{消令如泥,亦有}\\
  \z{合為漿者。凡服玉,皆不得用已成器物,及塚中玉璞也。好玉出藍田,及南陽徐善亭部界中,日南、盧容水中,外}\\
  \z{國於闐、疎勒諸處皆善。《仙方》名玉為玄真,潔白如豬膏,叩之鳴者,是真也。其比類甚多相似,宜精別之。所以燕}\\
  \z{石入笥}\f{<a>}\z{﹝二﹞,}\f{</a>}\z{卞氏長號也。(《新修》三頁,《大觀》}\z{卷三),《政和》八一頁)}\f{</sm></p>}
\end{typeChinese}

\section{Footnote text}
\begin{mainrule}
  Type each footnote text where it appears, surrounded by §<fn>§ and
  §</fn>§. Start a new line for every new footnote.
\end{mainrule}
\vspace{3mm}

%% tell them also to encode the heading of the footnote with <h4>
\begin{tabular}{@{}ll}
\parbox[b]{131mm}{
  \begin{typeChinese}
    \f{<h4>}\z{注}\f{</h4>}\\
    \f{<fn>}\z{﹝一﹞輩 《綱目》作「浸」。}\f{</fn>}\\
    \f{<fn>}\z{﹝二﹞笥  玄《大觀》作「筒」。}\f{</fn>}
  \end{typeChinese}
} & 
\includegraphics[height=4cm]{image2}
\end{tabular}


\section{Kaiti \chin{楷體}}
\begin{mainrule}
  Please surround the text in kaiti (\chin{楷體}) font as shown on the image with the tags §<k>§ and §</k>§.
\end{mainrule}
\vspace{3mm}

\begin{sampleImageSmall}[\ 3: \, Kaiti \chin{楷體}]{height=10cm}{image3.png}
\hspace{-25mm}
  \begin{typeChinese}
 \f{<h3><k>}\z{玉泉}\f{</k></h3>}\\
 \f{<p><k>}\z{味 甘,平,}\f{</k>}\z{無}\untranscribedText\z{。}\f{<k>}\z{主治}\f{<a>}\z{﹝一﹞}\f{</a>}\z{五}\f{<a>}\z{﹝二﹞}\f{</a>}\z{藏}\untranscribedText\z{柔}\f{<a>}\z{﹝三﹞}\f{</a>}\z{筋}\untranscribedText\z{魄}\f{<a>}\z{﹝四﹞}\f{</a>}\z{,}\\
\someText \f{</p>}
  \end{typeChinese}
  %% shortened the example a bit to make it fit on the page
%% \f{<p><k>}\z{味 甘,平,}\f{</k>}\z{無毒。}\f{<k>}\z{主治}\f{<a>}\z{﹝一﹞}\f{</a>}\z{五}\f{<a>}\z{﹝二﹞}\f{</a>}\z{藏百病,柔}\f{<a>}\z{﹝三﹞}\f{</a>}\z{筋強骨,安魂魄}\f{<a>}\z{﹝四﹞}\f{</a>}\z{,}\\
\end{sampleImageSmall}


\section{Itemized Lists}
\begin{mainrule}
  The book contains also itemized lists as depicted below. Items start
  either with the character "●" (U+25CF) or "○" (U+25CB) or with no character at all.
  Please use the character sequence §" # "§ as a delimiter between items.
\end{mainrule}
\vspace{3mm}

\begin{sampleImageSmall}[\ 4: \, Itemized lists]{height=10cm}{image4.png}
  \begin{typeChinese}
    \f{<h3>}\z{傷寒}\f{</h3>}\\
    \f{<list>}\\
    \z{○麻黃}\f{ # }\z{葛根}\f{ # }\z{○杏人}\f{ # }\z{茈胡}\f{<a>}\z{﹝一﹞}\f{</a>}\f{ # }\z{前胡}\f{ # }\z{●大青}\f{ # }\z{●龍膽}\f{ # }\z{芍<a>﹝二﹞</a>藥}\f{ # }\z{薰草}\\
    \f{ # }\z{升麻}\f{ # }\z{●牡丹}\f{ # }\z{○虎掌}\f{ # }\z{○朮}\f{ # }\z{防己}\f{ # }\z{●石膏}\f{ # }\z{牡蠣}\f{<a>}\z{﹝三﹞}\f{</a>}\f{ # }\z{貝齒}\f{<a>}\z{﹝四﹞}\f{</a> # }\z{鱉甲}\\
    \f{ # }\z{●犀角}\f{ # }\z{●零}\f{<a>}\z{﹝五﹞}\f{</a>}\z{羊角}\f{ # }\z{蔥白}\f{ # }\z{○生薑}\f{ # }\z{●豉}\f{ # }\z{●溺}\f{<a>}\z{﹝六﹞}\f{</a>}\f{ # }\z{●芒消}\\
    \f{</list>}
  \end{typeChinese}
\end{sampleImageSmall}

\end{document}

%%% Local Variables: 
%%% mode: latex
%%% TeX-master: t
%%% End: 
