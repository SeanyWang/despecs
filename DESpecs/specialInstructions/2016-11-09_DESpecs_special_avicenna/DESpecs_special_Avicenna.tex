%!TEX TS-program = xelatex 
%!TEX encoding = UTF-8 Unicode 

\documentclass[fontsize=11pt, paper=a4, 
DIV15,
normalheadings,
parskip=half-, 
pointlessnumbers]{scrartcl}

\usepackage[british]{babel} 

\usepackage{fontspec,xltxtra,xunicode} 
\defaultfontfeatures{Mapping=tex-text} 

\setromanfont[Mapping=tex-text]{DejaVu Serif}
\setsansfont[Scale=MatchLowercase,Mapping=tex-text]{Helvetica} 
\setmonofont[Scale=1.0]{Courier New} 

\frenchspacing

\usepackage{graphicx}
\graphicspath{{../../bilder/}}

\usepackage{longtable}

\usepackage{philokalia}

\usepackage{yfonts}

%%%

\input{../../../abbreviations/abbreviations_2}

%\usepackage{blacklettert1}

\begin{document}

\begin{center}
{\fontspec{Helvetica}{\LARGE \textbf{
Special Instructions for Hippocrates
}}} \\[5mm]
\large Klaus Thoden

\normalsize Max Planck Institute for the History of Science, Berlin, Germany

\today
\end{center}

%\tableofcontents

\section{Column numbers in Avicenna}

The major part of the text is printed in two columns per page. There
are no page numbers in this book. Instead, each column has a number in
the running header of the page. Please proceed as follows:

\begin{mainrule}
Begin each page with a §<pb>§ (without page number), followed by the
running head §<rh>…</rh>§. Include the number of the column in the
column tag.
\end{mainrule}

\begin{clarification}
This is a deviation of the rule given in section 2.3 of the
main DE Specs document, where the counting of the columns starts at 1 on every
new page.
\end{clarification}

\begin{sampleImage}[: Continuous column numbering]{normal}
\begin{typeLatin}
\bold{<pb>}\bold{<rh>}Sanctorij Commentaria\bold{</rh>} \\
\bold{<col 641>} \\
cerebri non proueniunt à cerebro per motum,\\
verl per aliquod aliud in$trumentum, $ed imme-\\
diatè exercentur in ip$amet cerebri $ub$tantia:\\
\untranscribedText\\
\bold{</col>} \\ 
\bold{<col 642>} \\
ru$e, ide$t recipientia facultatem à tribus parti-\\
bus principatum obtinentibus, videlicet à cor-\\
de, cerebro, \& iecore, quod videtur aduer$ari do\\
\untranscribedText\\
\bold{</col>} \\ 

\end{typeLatin}
\end{sampleImage}

\section{Headline across both columns}
\label{sec:headline-across-both}

In two cases (§0210.jpg§ and §0359.jpg§) there is a large headline spanning both columns. Please key those pages as follows:

\begin{mainrule}
  If a headline spans both columns, type both columns of the headline
  first, including the column numbers in the §<col>§ tag. Then, type
  the headline followed by the two columns below it and again include
  the column numbers you find on the top of the page.
\end{mainrule}

\begin{sampleImage}[: Column-spanning headline]{spanheading}
  \begin{typeLatin}
\bold{<pb>}\bold{<rh>}In Primam Fen.\bold{</rh>} \\
\bold{<col 687>} \\
poris redditur debilis, itaut eius expultrix no-\\
xium quod demandatur nõ po$$it expellere, ad\\
illam, tanquam ad excrementorum cloacã om-\\
\untranscribedText\\
\bold{</col>} \\ 
\bold{<col 648>} \\
omni\~u o$$ium, mu$culor\~u, neruorum, venarum\\
\& arteriarum: quæ con$iderato cum pertineat\\
ad anatomiam, illam con$ultò omittimus: quia\\
\untranscribedText\\
\bold{</col>} \\ 
\bold{<h>}DOCTRINA SEXTA\\
CAPVT PRIMVM.\\
De facultatibus, $eu virtutibus in genere.\bold{</h>}\\
\bold{<col 687>} \\
\bold{<p>}_VIrtitum, \& operationum ad inui-\\
cero aliæ ex alijs cogno$c\~utur: om-\\
nis enim virtus operationis, exi$tit\\
initium: nec aliqua operatio pro-\\
uenit, ni$i ex virtute: ideoque\\
utra$que in vno po$uimus capitulo_\bold{</p>}\\
\untranscribedText\\
\bold{</col>} \\ 
\bold{<col 648>} \\
7. metaph. 23. docentem exemplo artis edifica-\\
toriœ, omnes artes docentes tradendas e$$e ordi-\\
ne re$olutio, \& non compo$itio.\bold{</p>}\\
\untranscribedText\\
\bold{</col>} \\ 
    
  \end{typeLatin}
\end{sampleImage}
\section{Letters between columns}
\label{sec:lett-betw-columns}
Between the columns on each page, letters >>A<<, >>B<<, >>C<<, >>D<< and >>E<< are printed.

\begin{mainrule}
Do not type the letters between the columns.
\end{mainrule}

\end{document}
