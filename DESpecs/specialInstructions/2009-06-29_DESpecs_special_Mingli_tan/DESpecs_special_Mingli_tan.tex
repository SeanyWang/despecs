%!TEX TS-program = xelatex 
%!TEX encoding = UTF-8 Unicode 

\documentclass[fontsize=11pt, paper=a4, 
DIV15,
headings=normal,
parskip=half-, 
numbers=noenddot]{scrartcl}

\usepackage[british]{babel} 

\usepackage{fontspec,xltxtra,xunicode} 
\defaultfontfeatures{Mapping=tex-text} 

\setromanfont[Mapping=tex-text]{DejaVu Serif}
\setsansfont[Scale=MatchLowercase,Mapping=tex-text]{Helvetica} 
\setmonofont[Scale=1.0]{Courier New} 

\frenchspacing

\usepackage{graphicx}
\graphicspath{{../../Bilder/}}

\usepackage{longtable}

%%%

\input{../../abbreviations/abbreviations_2}

\newcommand{\hash}{{\char"0023}}
\newenvironment{typeChinese}{\begin{alltt}\s\begin{tabular}{@{}l}}{\end{tabular}\end{alltt}}

\newcommand{\chin}[1]{{\fontspec{Sun-ExtA}{#1}}}
\newcommand{\sunExtA}[1]{{\fontspec{Sun-ExtA}{#1}}}
\newcommand{\sunExtB}[1]{{\fontspec{Sun-ExtB}{#1}}}

\newcommand{\mincho}[1]{{\fontspec{MS Mincho}{#1}}}
\newcommand{\hira}[1]{{\fontspec{HiraMinPro-W3}{#1}}}

% Verkürzung von \bold und \chin:
% schon vorhanden: ch, c, s;  b, bf
% noch nicht vorhanden: sun (bringt aber nicht viel)
\newcommand{\f}[1]{\bold{#1}} % f für fett
\newcommand{\z}[1]{\chin{#1}} % z für Zeichen

%%%

\begin{document}

\begin{center}
{\fontspec{Helvetica}{\LARGE \textbf{
Special Instructions for Mingli tan \z{名理探}
\\[3mm]
(Addendum to DESpecs for Chinese text 2.0.1) 
}}} \\[5mm]
\large Wolfgang Schmidle, Martina Siebert, Joachim Kurtz

\normalsize Max Planck Institute for the History of Science, Berlin, Germany

\today
\end{center}

\section{Punctuation Marks}
\label{punctuation marks}

\begin{mainrule}
Type the filled circle as “fullwidth full stop” \z{.} (U+FF0E). 
Type the filled double circle as “fullwidth colon” \z{:} (U+FF1A).
%Type the comma \z{、} as “fullwidth comma” \z{,} (U+FF0C).
\end{mainrule}

\begin{clarification}
Type the ideographic full stop \z{。} (U+3002) and the ideographic comma \z{、} (U+3001) as you always do.
% Type the ideographic full stop \z{。} (U+3002) as you always do.
\end{clarification}

\vspace{8mm}
\begin{tabular}{@{}ll}
\parbox[b]{114mm}{
\htsc{Example} \\[100mm]
} & 
\includegraphics[height=5cm]{Punctuation_Variants_p9.pdf}
\end{tabular}

\vspace{-12mm}

\begin{typeChinese}
\f{<pb} \z{一}\f{><rh>}\z{名理探}\f{<sm>}\z{卷之一}\f{</sm></rh>} \\
\f{<h 2>}\z{愛}\f{<}\z{知}\f{2>}\z{學原始}\f{</h>} \\
\z{愛}\f{<}\z{知}\f{2>}\z{學者、西云}\f{<wl>}\z{斐錄瑣費亞}\f{</wl>}\z{、乃窮理諸學之}\f{<}\z{總}\f{R>}\z{名.譯名、則}\f{<}\z{知}\f{2>} \\
\z{之嗜.譯義、則言}\f{<}\z{知}\f{2>}\z{也。古有國王問於大賢人曰:汝深於}\f{<}\z{知}\f{2>}\z{、吾} \\
\z{夙聞之、不知何種之學爲深。對曰:余非能}\f{<}\z{知}\f{2>}\z{、惟愛}\f{<}\z{知}\f{2>}\z{耳。後賢} \\
\z{學}\f{<}\z{務}\f{V>}\z{辟}\f{<}\z{傲}\f{V>}\z{、故不敢用}\f{<}\z{知}\f{2>}\z{者之名、而第取愛}\f{<}\z{知}\f{2>}\z{爲名也。古稱大} \\
\f{<}\z{知}\f{2>}\z{者三人:一、}\f{<sl>}\z{索加德 }\f{</sl>}\z{.一、}\f{<sl>}\z{霸辣篤}\f{</sl>}\z{.一、}\f{<sl>}\z{亞利斯多特勒}\f{</sl>}\z{.}\f{<sl>}\z{亞利}\f{</sl>}\z{學問} \\
\end{typeChinese}


\begin{crossref}
For §< 2>§ see \sect{readings}.
\end{crossref}

\begin{note}
Some running heads contain signatures, for example the “1” in the lower left corner of this page. Do not type these signatures.
\end{note}


\section{Indications of Variant Readings}
\label{readings}

\begin{mainrule}
A circle in the corner of a character is marked by §< 1>§ (upper left corner), §< 2>§ (upper right corner), §< 3>§ (lower left corner) or §< 4>§ (lower right corner).
\end{mainrule}

\begin{clarification}
Unlike §< V>§ or §< R>§, mark every occurrence of the circle.
\end{clarification}

\begin{crossref}
Examples can be found in sections \ref{punctuation marks} and \ref{toc}.
\end{crossref}

\vspace{3mm}
\begin{note}
We number the corners according to the Four Corner Index (\z{四角號碼}). We do not attempt to mimic any historic numbering; for example, the §< 2>§ in the examples in sections \ref{punctuation marks} and \ref{toc} indicates the fourth tone.
\end{note}


\section{Table of Contents}
\label{toc}

\begin{mainrule}
Mark the lines in the table of contents as headings.
\end{mainrule}

\begin{clarification}
Type the pagenumbers after the §</h>§, separated by a single ideographic space U+3000.
\end{clarification}

\vspace{5mm}
\begin{tabular}{@{}ll}
\parbox[b]{111mm}{
\htsc{Example} \\[55mm]
\begin{typeChinese}
\f{<toc>} \\
... \\
\f{<ti>}\z{卷之二}\f{</ti>} \\
\f{<ti>}\z{五分稱之解}\f{</ti>}\z{ 一} \\
\f{<h 1>}\z{〇五公之篇第一}\f{</h>}\z{ 一} \\
\f{<h 2>}\z{立公稱者何義辯一}\f{<sm><}\z{隨}\f{V>}\z{論三}\f{</sm></h>}\z{ 三} \\
\f{<h 3>}\z{公者非虛名}\f{<}\z{相}\f{2><sm>}\z{一}\f{</sm></h>}\z{ 五} \\
\f{<h 3>}\z{公性不別於賾而自立}\f{<sm>}\z{二}\f{</sm></h>}\z{ 十四} \\
\f{<h 3>}\z{公性正解}\f{<sm>}\z{三}\f{</sm></h>}\z{ 十八} \\
... \\
\f{</toc>} \\[5mm]
\end{typeChinese}
} & 
%\includegraphics[height=6cm]{ToC_Levels_p6.pdf}
\includegraphics[height=12cm]{Mingli_p6_mit_Seitenzahlen.pdf}
\end{tabular}


\section{Specifically Marked Passages}

\begin{mainrule}
If a single character is in parentheses, type §( )§, i.e. ASCII parentheses U+0028 and U+0029, around the character.
\end{mainrule}

\begin{clarification}
Do not mark the smaller size of this character.
\end{clarification}

\vspace{8mm}
\begin{tabular}{@{}ll}
\parbox[b]{113.5mm}{
\htsc{Example} \\[20mm]
\begin{typeChinese}
\f{<pb><rh>}\z{名理探}\f{<sm>}\z{卷之二}\f{</sm></rh>} \\
\f{<p>(}\z{古}\f{)}\z{欲徹十倫府.解釋凡物} [...] \\
\z{五公稱.宗類殊獨依.}\f{<}\z{約}\f{R>}\z{拈} [...] \\
\z{中亦寓奧理.或實在於物.} [...] \\
\z{妙理.或別倫屬立.或依可} [...] \\
\f{(}\z{解}\f{)<sl>}\z{薄斐畧}\f{</sl>}\z{弟子曰}\f{<sl>}\z{計洒}\f{</sl>}\z{者、} [...] \\
\end{typeChinese}
} & 
\includegraphics[height=5cm]{HeadCharacters_p51.pdf}
\end{tabular}


\end{document}
