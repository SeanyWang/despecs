%!TEX TS-program = xelatex 
%!TEX encoding = UTF-8 Unicode 

\documentclass[fontsize=11pt, paper=a4, 
DIV15,
headings=normal,
parskip=half-, 
numbers=noenddot]{scrartcl}

\usepackage[british]{babel} 

\usepackage{fontspec,xltxtra,xunicode} 
\defaultfontfeatures{Mapping=tex-text} 

\setromanfont[Mapping=tex-text]{DejaVu Serif}
\setsansfont[Scale=MatchLowercase,Mapping=tex-text]{Helvetica} 
\setmonofont[Scale=1.0]{Courier New} 

\frenchspacing

\usepackage{graphicx}
\graphicspath{{../../Bilder/}}

\usepackage{longtable}

%%%

\input{../../abbreviations/abbreviations_2}

\newcommand{\hash}{{\char"0023}}
%\newenvironment{typeChinese}{\begin{alltt}\s\begin{tabular}{@{}l}}{\end{tabular}\end{alltt}}
\newenvironment{typeChinese}{\begin{alltt}\large\begin{tabular}{@{}l}}{\end{tabular}\end{alltt}} % nicht small !

\newcommand{\chin}[1]{{\fontspec{Sun-ExtA}{#1}}}
\newcommand{\sunExtA}[1]{{\fontspec{Sun-ExtA}{#1}}}
\newcommand{\sunExtB}[1]{{\fontspec{Sun-ExtB}{#1}}}

\newcommand{\mincho}[1]{{\fontspec{MS Mincho}{#1}}}
\newcommand{\hira}[1]{{\fontspec{HiraMinPro-W3}{#1}}}

% Verkürzung von \bold und \z:
% schon vorhanden: ch, c, s;  b, bf
% noch nicht vorhanden: sun (bringt aber nicht viel)
\newcommand{\f}[1]{\bold{#1}} % f für fett
\newcommand{\z}[1]{\chin{#1}} % z für Zeichen

%%%

\begin{document}

\begin{center}
{\fontspec{Helvetica}{\LARGE \textbf{
Special Instructions for \\ ``Heidelberg Enzyklopädien Projekt'' 
\\[3mm]
(Addendum to DESpecs for Chinese text 2.1) 
}}} \\[5mm]
\large Cathleen Päthe $^\textrm{\footnotesize a}$, Wolfgang Schmidle $^\textrm{\footnotesize b}$, Martina Siebert $^\textrm{\footnotesize c}$

\normalsize 


\normalsize 
$^\textrm{\footnotesize a}$ Cluster ``Asia and Europe'', University of Heidelberg, Germany \\
$^\textrm{\footnotesize b}$ Max Planck Institute for the History of Science, Berlin, Germany \\
$^\textrm{\footnotesize c}$ Berlin State Library, Prussian Cultural Heritage Foundation, Berlin, Germany

\today
\end{center}

%\tableofcontents

%\section{Bibliographical Information}
%Start a new text file for each encyclopedia. At the beginning of the file, please type all the bibliographical information that is given in the text. 
%\vspace{3mm}
%\begin{example}%[ 1]
%...
%\end{example}


%\section{Character Variants}
%Do not mark a character variant if only the radical is different, i.e. do not use §< R>§. However, please use §< V>§ for other variants (see e.g. example 2).

\vspace{1cm}
Please do NOT type the complete texts. Instead, please type for each encyclopedia:
\begin{enumerate}
\item The table of contents, if there is one
\item The entries specified by the attached list. 
\end{enumerate}

\section{Tables of Contents}

Please type the tables of contents according to the rules in the DESpecs. In addition, if you can identify the heading levels, please use §<h 1>§, §<h 2>§ and so on. 

\vspace{2mm}
\begin{example}

\vspace{-5mm}
\begin{typeChinese}
\f{<ti>}\z{天工開物卷目錄}\f{</ti>} \\
\f{<h 1>}\z{上卷}\f{</h>} \\
\f{<h 2>}\z{乃粒第一卷}\f{</h>} \\
\f{<h 2>}\z{乃服第二卷}\f{</h>} \\
\f{<h 2>}\z{彰施第三卷}\f{</h>} \\
... \\
\f{<h 1>}\z{中卷}\f{</h>} \\
\f{<h 2>}\z{陶埏第七卷}\f{</h>} \\
... \\
\end{typeChinese}

\end{example}

\section{Typing Encyclopedia Entries}


%\begin{mainrule}
%\end{mainrule}
%
%\begin{clarification}
%\end{clarification}
%
%\begin{itemize}
%\item 
%\end{itemize}

% EXAMPLE Liste

Mark the pages and columns of the typed entries with §<pb>§ and §<col>§. Do not use §</col>§. Do not mark the pages and columns that do not belong to a typed entry.

Please mark the entry head with §<h> </h>§. Mark the translations with a new tag §<tr> </tr>§ on a separate line. 

Circled characters are marked with §<num> </num>§; see example 2.

\vspace{3mm}
\begin{example}[ 1]

\begin{tabular}{|l|l|l|l|}
\hline
Book Title & File & Entry & Pages \\[1mm]
\hline
\hline
She hui wen ti ci dian \z{社會問題辭典} & §swc_550-649.pdf§ & \z{國家} & 557–559 \\[1mm]
\hline
\end{tabular}

\begin{typeChinese}
\f{<pb} 557\f{><rh>}\z{十一畫 國}\f{</rh>} \\
\f{<col 2>} \\
\f{<h>}\z{國家}\f{</h>} \\
\f{<tr>}State, \z{德} Staat\f{</tr>} \\
\f{<p i>}[\f{<}\z{概}\f{V>}\z{說}]\z{ 凡在一定的國土上,有}\f{<}\z{統}\f{R>}\z{治}\f{<}\z{組}\f{R>}\f{<}\z{織}\f{R>}\z{的人類} \\
\z{集合}\f{<}\z{體}\f{V>}\z{,謂之國家。土地、人民、政治是成立國家的} \\
\z{三要素,三者缺其一,不得稱爲國家。其中尤以統治關} \\
\z{係爲最重要,卽一階}\f{<}\z{級}\f{V>}\z{沒有對於其他階級的支配關} \\
\z{係,不得成立爲國家。元來以上對於國家的}\f{<}\z{解}\f{V>}\z{釋,自}\f{<}\z{近}\f{V>} \\
\z{代成立完備的國家組織後才有的,然國家的意義,}\f{<}\z{隨}\f{V>} \\
\z{時代的推移,也有許多}\f{<}\z{變}\f{V>}\f{<}\z{遷}\f{V>}\z{。}\f{</p>} \\
\f{<p i><dl>}\z{歐洲}\f{</dl>}\z{古代的政治圑體只限於都市。例如}\f{<dl>}\z{希臘}\f{</dl>}\z{人對} \\
\z{於國家使用} Polis \z{(市)的文字,}\f{<dl>}\z{}\f{<}\z{羅}\f{R>}\z{馬}\f{</dl>}\z{人使用} Civitas \\
\z{或} respublica \z{的文字,可知古代國家明明是以都市} \\
... \\
\f{<pb} 558\f{><rh>} ... \f{</rh>} \\
\f{<col 1>} \\
... \\
... \f{</p>} \\
\f{<p i>}[\z{發達}]\z{ 關於國家的}\f{<}\z{起}\f{V>}\z{源,除開}\f{<cl>}\z{神意說契}\f{<}\z{約}\f{V>}\z{說}\f{</cl>}\z{外,} \\
\f{<col 2>} \\
\z{以社會學的見地作根據,有}\f{<dl>}\z{家族起源說}\f{</dl>}\z{與}\f{<dl>}\z{征服起源} \\
\z{說}\f{</dl>}\z{兩種。} ... \\
... \\
\f{<pb} 559\f{><rh>} ... \f{</rh>} \\
\f{<col 1>} \\
... \\
\z{了。}\f{</p>} \\
\end{typeChinese}

\end{example}

\newpage
\begin{example}[ 2]

\begin{tabular}{|l|l|l|l|}
\hline
Book Title & File & Entry & Pages \\[1mm]
\hline
\hline
Zhe xue ci dian \z{哲學辭典} & §zxcd_11_11hua.pdf§ & \z{教育} & 606–607 \\[1mm]
\hline
\end{tabular}

\begin{typeChinese}
\f{<pb} 606\f{><rh>}\z{十一畫 教}\f{</rh>} \\
\f{<col 1>} \\
\f{<h>}\z{教育}\f{</h>} \\
\f{<tr>}\z{英} Education \\
\z{法} Education \\
\z{德} Erziehung\f{</tr>} \\
\f{<p>}\z{原語由}\f{<sl>}\z{拉丁}\f{</sl>}\z{之} Educare \z{而來。本誼爲引出。言引}\f{<}\z{導}\f{V>} \\
\z{兒童之固有性能,使之完全發展也。此語用法,其範圍} \\
\z{廣狹不同。}\f{<num>}\z{甲}\f{</num>}\z{廣而言之,凡足}\f{<}\z{感}\f{R>}\z{化身心之影}\f{<}\z{響}\f{R>}\z{,俱得云} \\
\z{教育。只稱其}\f{<}\z{結}\f{R>}\z{果,不計其方法。如曰家庭教育是。是不} \\
\z{言教育,而自寓教育之用在內者也。}\f{<num>}\z{乙}\f{</num>}\z{狹而言之,則惟} \\
\z{具有目的,出以一定方案者,始云教育。此中亦分二類。} \\
\z{(其一)則對象及期限有定,其功效又可以明確表出} \\
\z{者。(其二)反是。前者指學校教育。後者指社會教育。}\f{</p>} \\
\f{<p>} ... \\
... \\
\f{<col 2>} \\
... \\
\f{<pb} 607\f{><rh>} ... \f{</rh>} \\
\f{<col 1>} \\
... \\
\z{其理想。}\f{</p>} \\
\end{typeChinese}

\end{example}

\begin{note}
Please keep in mind to use the full range of Unicode 5.1 (but not 6.0). For example, please type variants such as \z{爲} instead of \z{為} and \z{卽} instead of \z{即}.

Note that if a variant is not in Unicode, you only have to mark variants once per encyclopedia. For instance, the variant of \z{統} in example 1 needs to be marked only once. However, if it also occurs in an entry of \z{哲學辭典}, it needs to be marked once again.

Also note that both examples are treated as if they were the beginning of a text, i.e. character variants are marked even if they have already appeared on earlier pages of the same encyclopedia. 
\end{note}

\end{document}
