%!TEX TS-program = xelatex 
%!TEX encoding = UTF-8 Unicode 

\documentclass[fontsize=11pt, paper=a4, 
DIV15,
normalheadings,
parskip=half-, 
pointlessnumbers]{scrartcl}

\usepackage[british]{babel} 

\usepackage{fontspec,xltxtra,xunicode} 
\defaultfontfeatures{Mapping=tex-text} 

\setromanfont[Mapping=tex-text]{DejaVu Serif}
\setsansfont[Scale=MatchLowercase,Mapping=tex-text]{Helvetica} 
\setmonofont[Scale=1.0]{Courier New} 

\frenchspacing

\usepackage{graphicx}
\graphicspath{{./Bilder/}}

\usepackage{longtable}

\usepackage{philokalia}

%%%

%!TEX TS-program = xelatex
%!TEX encoding = UTF-8 Unicode

\usepackage{xspace} \xspaceaddexceptions{”}
\usepackage{ifthen}

\newcommand{\ch}[1]		{chapter~\ref{#1}}
\newcommand{\sect}[1]		{section~\ref{#1}}
\newcommand{\fig}[1]		{figure~\vref{#1}}
\newcommand{\Fig}[1]		{Figure~\vref{#1}}
\newcommand{\tbl}[1]		{table~\vref{#1}}

\newcommand{\ie}			{i.e. }
\newcommand{\eg}			{e.g. }

\newcommand{\qq}			{\qquad}

\newcommand{\spitz}[1]		{\ensuremath{\langle}#1\ensuremath{\rangle}}

\newcommand{\mehrzeilen}[1][1]{\enlargethispage{#1\baselineskip}}


% Zeilenabstand

\usepackage{setspace}
\newenvironment{enum}{\begin{enumerate} \singlespacing} {\end{enumerate}}
\newenvironment{items}{\begin{itemize} \singlespacing} {\end{itemize}}


% verbatim

\usepackage{verbatim}

\usepackage{alltt}
\newcommand{\klein}{\small}
\newenvironment{exakt}[1][\small]{\singlespacing#1\begin{alltt}}{\end{alltt}}

\usepackage{shortvrb}
\MakeShortVerb{\§}


%%%%%%%%%%%%%%%%%%%%%%%%%%%%%%%%

% xml commands
% use for any xml markup, brackets supplied
\newcommand{\xml}[1]{§<#1>§}
% use for milestone tags
\newcommand{\xms}[1]{§<#1/>§}
% closing element markup
\newcommand{\xmcl}[1]{§</#1>§}
% full xml markup example
\newcommand{\xmex}[1]{§#1§}
% attribute markup, the first argument is the element name
\newcommand{\attr}[2]{§@#2§}

\newcommand{\bold}{\textbf}
% ligature
\newcommand{\li}[1]{\bold{\{}#1\bold{\}}}

\newcommand{\xs}{\scriptsize}
\newcommand{\s}{\footnotesize}

%

\newcommand{\bs}{\textbackslash}
\newcommand{\tld}{\textasciitilde}

\newcommand{\tocspace}{\addtocontents{toc}{\protect\vspace{1mm}}}

\newcommand{\unicode}[1]{{\fontspec{Apple Symbols}{\Large #1}}}
\newcommand{\§}{{\char"00A7}}

%%%%%%%%%%%%%%%%

\newcommand{\htsc}[1]{\emph{#1}}
\newcommand{\lig}[1]{\fontspec{Hoefler Text}{\Large #1}}
\newcommand{\fraktur}[1]{{\fontspec{BreitkopfFraktur}{\LARGE #1}}}

%

\newenvironment{mainrule}{}{}
\newenvironment{mainruleLessImportant}{}{}
\newenvironment{clarification}{\s}{}
\newenvironment{exception}{\htsc{Exception:}}{}
\newenvironment{note}{\textbf{Please note:}}{}
\newenvironment{crossref}{\s\ensuremath{\longrightarrow}}{}

%

\newenvironment{sampleImage}[2][]{\parbox{\linewidth}{{\htsc{Example#1}} \\[3mm] \includegraphics[width=\linewidth]{#2}}}{}
\newenvironment{sampleImageSmall}[3][]{\parbox{\linewidth}{{\htsc{Example#1}} \\[3mm] \includegraphics[#2]{#3}}}{}

\newenvironment{example}[1][]{\htsc{Example#1} \\}{}
\newenvironment{exampleTest}[2][]{\parbox{\linewidth}{\htsc{Example #1} \\[3mm] #2}}{} % ??

\newenvironment{liste}[1][]{\htsc{List#1} \\}{}
\newenvironment{tabelle}[1][]{\htsc{Table#1} \\}{}

%

\newenvironment{typeLatin}{\begin{alltt}\s\begin{tabular}{@{}l}}{\end{tabular}\end{alltt}}

\newfontfamily{\greek}[Scale=0.95]{Courier New}
\newenvironment{typeGreek}{\begin{alltt}\greek\s\begin{tabular}{@{}l}}{\end{tabular}\end{alltt}}

\newenvironment{typeMath}{\begin{alltt}\begin{tabular}{l}}{\end{tabular}\end{alltt}}

%

\newfontfamily{\muh}[Scale=0.9]{DejaVu Serif}
\newcommand{\someText}{...} % {{\muh\textit{(some text)}}}
\newcommand{\untranscribedText}{...} % {{\muh\textit{(some untranscribed text)}}}
\newcommand{\notTranscribed}{{\muh\textit{(not transcribed)}}}
\newcommand{\missingText}[1]{{\muh\textit{(#1)}}}

%% Chinese bits
\newenvironment{typeChinese}{\begin{alltt}\s\begin{tabular}{@{}l}}{\end{tabular}\end{alltt}}

\newcommand{\chin}[1]{{\fontspec{Sun-ExtA}{#1}}}
\newcommand{\sunExtA}[1]{{\fontspec{Sun-ExtA}{#1}}}
\newcommand{\sunExtB}[1]{{\fontspec{Sun-ExtB}{#1}}}

\newcommand{\mincho}[1]{{\fontspec{MS Mincho}{#1}}}
\newcommand{\hira}[1]{{\fontspec{HiraMinPro-W3}{#1}}}

\newcommand{\f}[1]{\bold{#1}} % f für fett
\newcommand{\z}[1]{\chin{#1}} % z für Zeichen


\begin{document}

\begin{center}
{\fontspec{Helvetica}{\LARGE \textbf{
Special Instructions for Work Order V
\\[3mm]
(Addendum to Data Entry Specs 1.1.2) 
}}} \\[5mm]
\large Wolfgang Schmidle, Klaus Thoden, Malcolm D. Hyman

\normalsize Max Planck Institute for the History of Science, Berlin, Germany

\today
\end{center}

\tableofcontents


\section{Indexes and Tables of Contents}


\subsection{Indexes}

\begin{mainrule}
An index is marked by §<ind>§ and §</ind>§. Use §#§ for large spaces.
%, for example between text and reference. 
Type a return after each row. 
%If you can identify a table as an index, mark it by §<ind>§ and §</ind>§. Use §#§ as separator between text and reference.
\end{mainrule}

% Ob sie für jede Seite einen getrennten Index machen, sollen sie slebst entscheiden.


\begin{sampleImage}[1]{bacon_253}

\begin{typeLatin}
\bold{<ind it>} \\
Caterpillars \bold{#} \bold{_}153\bold{_} \\
Cements that grow hard \bold{#} \bold{_}183\bold{_} \\
Chalk, a good compo$t, \bold{_}122, 123\bold{_}. Good for \\
\bold{#} Pa$ture, as well as for Arable \bold{#} \bold{_}ibid\bold{_}. \\
Chameleons, \bold{_}80\bold{_}. Their nouri$hment, \bold{#} \bold{_}ibid\bold{_}. \\
\bold{#} A fond Tradition of them \bold{#} \bold{_}ibid\bold{_}. \\
\bold{</ind>} 
\end{typeLatin}
\end{sampleImage}


\begin{sampleImage}[2]{gallac_91}

\begin{typeLatin}
\bold{<ind>} \\
\bold{<col 1>} \\
\someText \\
Diligenz $overchia, quale. \bold{#} 49 \\
Diminuzione di gro$$ozze, come deb- \\
\bold{#} ba condur$i. \bold{#} 56 \\
\bold{_}Diocleziano\bold{_} . Sue Terme. \bold{#} 51 \\
\someText \\
\bold{</col>} \\
\bold{<col 2>} \\
\someText \\
Errori di que$to genere, cagione di \\
\bold{#} tutti gli errori. \bold{#} 18. 19 \\
\bold{#} Provvedimenti dei Romani con- \\
\bold{#} tro a que$ti errori. \bold{#} 19 \\
\someText \\
\bold{</col>} \\
\bold{</ind>} \\
\end{typeLatin}
\end{sampleImage}


\subsection{Tables of Contents}

\begin{mainrule}
A table of contents is marked by §<toc>§ and §</toc>§. Use §#§ for large spaces.
%, for example between section names and page numbers. 
Type a return after each row. 
%If you can identify a table as a table of contents, mark it by §<toc>§ and §</toc>§. Use §#§ as separator between section names and page numbers.
\end{mainrule}

%\begin{clarification}
%(How toc's can be recognized?)
%\end{clarification}

\begin{sampleImage}[1]{zubler_43_2}

\begin{typeLatin}
\bold{<toc it>} \\
Cap. 1. \bold{#} De Chorographia generatim: quid $it, & que ad eam In-\\
\bold{#} strumenta poti{$s}imùm requi$ita, \bold{#} pag 1. \\
II. \bold{#} De In$trumenti fabricâ, \bold{#} 2 \\
III. \bold{#} De Triangulis, omnium dimen$ionum fundamento, \bold{#} 5 \\
\someText \\
\bold{</toc>} 
\end{typeLatin}
\end{sampleImage}


\begin{sampleImage}[2]{belidor_683}

\begin{typeLatin}
\bold{<toc it>} \\
\bold{_}CH\bold{<sc>}APITRE\bold{</sc>} I.\bold{_} Où l'on en$eigne comme $e fait la pou$$ée des \\
\bold{#} Voutes, & où l'on raporte quelques principes tirés de la mé- \\
\bold{#} canique pour en faciliter l'intelligence \bold{#} 2 \\
\bold{_}C\bold{<sc>}HAP\bold{</sc>}. II. \bold{_}De la maniere de calculer l'épai$$eur des Pié-droits \\
\bold{#} des Voutes en plain ceintre pour e$tre en équilibre par leur ré- \\
\bold{#} $i$tance avec la pou$$ée qu'ils ont à $oútenir. \bold{#} 10 \\
\bold{</toc>} \\
\end{typeLatin}
\end{sampleImage}


\subsection{Other Structures With Large Spaces}

%Do not type dots or lines that only serve as placeholders. 

\begin{mainrule}
If a normal paragraph contains at least one large space, mark it by §#§, i.e. use §<p #>§. Mark each large space in the paragraph by §#§.
\end{mainrule}

\begin{clarification}
Before you use §<p #>§, make sure the paragraph is not part of a table, an index or a table of contents.
\end{clarification}

\begin{sampleImage}[2]{Pappus_large_spaces}

\begin{typeLatin}
\bold{<p #>} \\
\someText \\
extrema ad axes \bold{#} angulorum, continent autem hunc propo$itiones \\
ferè exi$tentes vna multa, & varia theoremata, & linearum, & $uperficie- \\
rum, & $olidorum omnia $imul vna demon$tratione, & quæ nondum de- \\
mon$trata $unt, & quæ \bold{#} & in duodecimo libro horum elemento- \\
\someText \\
\bold{</p>} \\
\end{typeLatin}
\end{sampleImage}





%Introduce a generic tag for leading: §<lead> # </lead>§ or so.

%%Does that make sense? Up to now, I did not explain leading, but they were supposed to grasp the concept through the examples. Here I would have to explain it. 

%Alternative: Section “Leading”, where leading is explained, with example. Then: toc's and indexes.

%One (weird) example would be Biancani 1635, p.195.

%Another semi-weird example: modern-style quotations. Alternatively in the block quotation section, or no rule at all.

%*

%Do we need a rule for lists? I guess not; in European texts it can either be typed as normal text, or it is a table, or it is a structure with leading. “Lists” would be a nice section title than “Other Structures With Leading”, though. But would it fit?


\section{Special Instructions for Individual Texts}

\subsection{Alberti (1565)}

\begin{mainrule}
Line numbers (§5§, §10§, §15§ and so on) are typed just like normal marginal notes, but are marked by §<ln> </ln>§.
\end{mainrule}

\begin{clarification}
The position of the line numbers to the left or right is not marked.
\end{clarification}

\subsection{Aristoteles (1548)}

\begin{mainrule}
Type the table with six columns on p.0205 as two separate tables, each with three columns.
\end{mainrule}

\subsection{Bion (1723)}

\begin{mainrule}
In the table on p.0209, do not type the vertically printed words. Type the last column (§La Regle ... $en$ible.§) as a normal paragraph outside the table.
\end{mainrule}


\subsection{Gravesande (1721/5)}

\begin{mainrule}
In the index starting on p.0805, type the horizontal lines as §#§.
\end{mainrule}


%\section{Questions}

%How many books can Formax type until the end of the year?

%Add additional rules about tables?

%Add additional rules about figures?



\end{document}
