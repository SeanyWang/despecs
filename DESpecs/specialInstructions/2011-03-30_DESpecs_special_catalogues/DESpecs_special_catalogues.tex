%!TEX TS-program = xelatex 
%!TEX encoding = UTF-8 Unicode 

\documentclass[fontsize=11pt, paper=a4, 
DIV15,
headings=normal,
parskip=half-, 
numbers=noenddot]{scrartcl}

\usepackage[british]{babel} 

\usepackage{fontspec,xltxtra,xunicode} 
\defaultfontfeatures{Mapping=tex-text} 

\setromanfont[Mapping=tex-text]{DejaVu Serif}
\setsansfont[Scale=MatchLowercase,Mapping=tex-text]{Helvetica} 
\setmonofont[Scale=1.0]{Courier New} 

\frenchspacing

\usepackage{graphicx}
\graphicspath{{../../Bilder/}}

\usepackage{longtable}

%%%

\input{../../abbreviations/abbreviations_2}

\newcommand{\hash}{{\char"0023}}
\newenvironment{typeChinese}{\begin{alltt}\begin{tabular}{@{}l}}{\end{tabular}\end{alltt}}
\newenvironment{typeChineseSmall}{\begin{alltt}\footnotesize\begin{tabular}{@{}l}}{\end{tabular}\end{alltt}}

\newcommand{\chin}[1]{{\fontspec{Sun-ExtA}{#1}}}
\newcommand{\sunExtA}[1]{{\fontspec{Sun-ExtA}{#1}}}
\newcommand{\sunExtB}[1]{{\fontspec{Sun-ExtB}{#1}}}

\newcommand{\mincho}[1]{{\fontspec{MS Mincho}{#1}}}
\newcommand{\hira}[1]{{\fontspec{HiraMinPro-W3}{#1}}}

% Verkürzung von \bold und \chin:
% schon vorhanden: ch, c, s;  b, bf
% noch nicht vorhanden: sun (bringt aber nicht viel)
\newcommand{\f}[1]{\bold{#1}} % f für fett
\newcommand{\z}[1]{\chin{#1}} % z für Zeichen

\newcommand{\smlb}{\bold{\textbackslash\textbackslash}}
%%%

\begin{document}

\begin{center}
{\fontspec{Helvetica}{\LARGE \textbf{
Special Instructions for Danshengtang
\\[3mm]
(Addendum to DESpecs for Chinese text 2.1) 
}}} \\[5mm]
\large Wolfgang Schmidle, Martina Siebert, Cathleen Päthe

\normalsize Max Planck Institute for the History of Science, Berlin, Germany

\today
\end{center}


\section{Collected Works \chin{澹生堂集}}


\begin{mainrule}
Type the handwritten underlinings as described in the DESpecs 2.1, page 15. In addition, use §<pl> </pl>§ for dotted lines.

However, please type single circles and dots as punctuation marks. If there is an underlining immediately before the punctuation mark, include the character (but not the punctuation mark).
\end{mainrule}

\begin{tabular}{@{}ll}
\parbox[b]{125mm}{
\htsc{Example } \\[17mm]
\begin{typeChinese}
... \bold{<pl>}\chin{其節短矣}\bold{</pl>}\chin{。旣猛烈而益} ... \\
... \chin{由是而別。使}\bold{<cl>}\chin{臨敵者氣}\bold{</cl>} ... \\
... \bold{<pl>}\chin{必固矣}\bold{</pl>}\chin{。}\bold{<pl>}\chin{然必車製合宜}\bold{</pl>}\chin{。} ... \\[12mm]
\end{typeChinese}
} & \qquad
\includegraphics[height=6cm]{Specs-Bild3}
\end{tabular}

\vspace{-7mm}
\begin{clarification}
(We have ignored the tone mark next to \chin{別} in this example.)
\end{clarification}

\section{Mantang \chin{漫堂鈔本} (\chin{澹生堂藏書目})}


\begin{mainrule}
This text is handwritten. Consequently, make use of the full range of Unicode, but do NOT mark character variants if they are not in Unicode. 
\end{mainrule}

\begin{clarification}
For example, the character \z{册} occurs as standard character and as a variant. Type both the standard character and the variant as \z{册}.
\end{clarification}

\begin{mainrule}
Do not type the running heads, neither from the facsimile nor from the original book. However, DO type the facsimile page numbers in square brackets.
\end{mainrule}

\begin{clarification}
If you can identify a single space within a name etc. as a decorative space, do not type it (see the DESpecs page 6).
\end{clarification}

\newpage
%\vspace{5mm}
\begin{example}[: page 0580]

\vspace{-5mm}
\begin{typeChineseSmall}

\f{<toc>} \\
\f{<pb a [}\z{五五八}\f{]>}\red{\f{<rh>}\z{續修四庫全書 史部 目錄類}\f{</rh>}} \\
\f{<h>}\z{圖說計六則}\f{</h>} \\
\f{<p>}\z{小學之目為爾雅為蒙書為家訓為纂訓為}\f{<001>}\z{學為}\f{</p>} \\
\f{<p>}\z{字學計六則}\f{</p>} \\
\f{<p>}\z{經部凡十二類共六十二目}\f{</p>} \\
\f{<pb c>} \\
\f{<h>}\z{澹生堂}\f{<002>}\z{書目}\f{(}\z{經部}\f{)</h>} \\
\f{<p>}\z{山陰祁氏}\f{<}\z{收}\f{R>}\z{集}\f{</p>} \\
\f{<p>}\z{易}\f{(}\z{古易}\f{\smlb}\z{圖譜}\f{)}\z{ }\f{(}\z{章句註傳}\f{\smlb}\z{拈}\f{<}\z{鮮}\f{V>}\z{考正}\f{)}\z{ }\f{(}\z{疏義集鮮}\f{\smlb}\z{識緯}\f{)}\z{ }\f{(}\z{詳說}\f{\smlb<004>}\z{易}\f{)}\z{ }\f{(}\z{蓍卜}\f{)</p>} \\
\f{<p>}\z{三墳}\f{(}\z{一}\f{<003>\smlb}\z{一冊}\f{)}\z{ 周易古文}\f{(}\z{二卷}\f{\smlb}\z{二冊}\f{)}\z{ }\f{(}\z{楊時喬}\f{)</p>} \\
\z{古易筌}\f{(}\z{二十九}\f{<003>\smlb}\z{鄧伯羔}\f{)}\z{ 周易古本}\f{(}\z{一卷}\f{\smlb}\z{一冊}\f{)}\z{ }\f{(}\z{羅夫竑}\f{)} \\
\z{古易}\f{<004>}\z{編}\f{(}\z{詞意集十三}\f{<003>}\z{ }\f{(}\z{象數集十三}\f{<003>}\z{ }\f{(}\z{虞仲}\f{\smlb}\z{翔等傳}\f{)}\z{ }\f{ (}\z{變占集二卷}\f{)}\z{ }\f{(}\z{共八冊}\f{)}\z{ }\f{(}\z{李本}\f{)</p>} \\
\f{<p i>(}\z{圖輯}\f{)</p>} \\
\f{<p>}\z{羲經十一翼}\f{(}\z{五卷}\f{)\smlb}\z{兆文 }\f{(}\z{六冊}\f{)}\z{ }\f{(}\z{巳上六種俱古易}\f{)}\z{ }\f{(}\z{傳}\f{)</p>} \\
\f{<p>}\z{周易真文}\f{(}\z{二卷}\f{)}\z{ 周易傳義}\f{(}\z{二十四卷}\f{\smlb}\z{冊}\f{)}\z{ }\f{(}\z{一套}\f{)}\z{ }\f{(}\z{八}\f{)</p>} \\
\f{<pb d>} \\
\f{<p>}\z{周易本義}\f{(}\z{四卷}\f{)}\z{ 周易本義通釋}\f{(}\z{十卷}\f{)}\z{ }\f{(}\z{四冊}\f{\smlb}\z{胡炳文}\f{)</p>} \\
\f{<p>}\z{易經程傳}\f{(}\z{十二}\f{\smlb}\z{卷}\f{)}\z{ }\f{(}\z{程顧}\f{)}\z{ 易經大全}\f{(}\z{十四卷}\f{)</p>} \\
\f{<p>}\z{易經旁訓}\f{(}\z{三卷}\f{)}\z{ 京氏易傳}\f{(}\z{三卷}\f{\smlb}\z{京房蓍}\f{)}\z{ }\f{(}\z{一冊}\f{)</p>} \\
\f{<p>}\z{周易例略}\f{(}\z{一卷}\f{\smlb}\z{一冊}\f{)}\z{ }\f{(}\z{王弼著}\f{\smlb}\z{邢註}\f{)}\z{ 鄭康成周易註}\f{(}\z{一}\f{<003>)</p>} \\
\f{<p>}\z{楊慈湖易傳}\f{(}\z{二十}\f{<003>\smlb}\z{一套}\f{)}\z{ }\f{(}\z{四冊}\f{\smlb}\z{宋楊簡}\f{)</p>} \\
\f{<p>}\z{楊誠齊易傳}\f{(}\z{二十}\f{<003>\smlb}\z{宋楊萬里}\f{)}\z{ }\f{(}\z{八冊}\f{)</p>} \\
\f{<p>}\z{今易銓}\f{(}\z{二十}\f{\smlb}\z{四卷}\f{)}\z{ }\f{(}\z{鄧伯羔}\f{)}\z{ 易鮮}\f{(}\z{一卷}\f{)}\z{ }\f{(}\z{載升庵雜録}\f{\smlb}\z{巳上俱章句注傳}\f{)</p>} \\
\f{<p>}\z{周易註疏}\f{(}\z{九卷}\f{\smlb}\z{八冊}\f{)}\z{ 周易鮮詁}\f{(}\z{三卷}\f{)</p>} \\
\f{<p>}\z{周易今文集說}\f{(}\z{九卷}\f{\smlb}\z{九冊}\f{)}\z{ }\f{(}\z{楊時喬}\f{)</p></toc>} \\
\f{<toc>} \\
\f{<pb a [}\z{五五九}\f{]>} \\
\f{<ti>}\z{澹生堂藏書目 經部}\f{</ti>} \\
\f{<p>}\z{周易集鮮}\f{(}\z{十卷}\f{\smlb<003>)}\z{ }\f{(}\z{五冊}\f{)}\z{ }\f{(}\z{唐李鼎祚}\f{)}\z{ }\f{(}\z{易鮮附録一}\f{)</p>} \\
\f{<p>}\z{易學四周}\f{(}\z{八}\f{<003>\smlb}\z{八冊}\f{)}\z{ }\f{(}\z{季本}\f{)</p>} \\
\f{<p>}\z{來}\f{<005>}\z{鮮易集註}\f{(}\z{十六}\f{<003>)}\z{ }\f{(}\z{十冊}\f{)}\z{ }\f{(}\z{來知德輯}\f{)</p>} \\
\f{<p>}\z{周易義海撮要}\f{(}\z{一}\f{<003>\smlb}\z{要}\f{)}\z{ }\f{(}\z{一冊}\f{)}\z{ }\f{(}\z{房審權編}\f{)}\z{ }\f{(}\z{李衡撮}\f{)</p>} \\
\f{<p><006>}\z{氏易鮮}\f{(}\z{八}\f{<003>\smlb}\z{四冊}\f{)}\z{ }\f{(}\z{蘇軾}\f{)}\z{ 東坡易傳}\f{(}\z{九卷}\f{\smlb}\z{一冊}\f{)</p>} \\
\f{<p>}\z{朱子易學啟蒙}\f{(}\z{四}\f{<003>)}\z{ }\f{(}\z{四冊}\f{)}\z{ }\f{(}\z{朱文公}\f{)</p>} \\
\f{<p>}\z{易學啟蒙集註}\f{(}\z{五}\f{<003>)}\z{ }\f{(}\z{五冊}\f{)}\z{ }\f{(}\z{楊時喬}\f{)</p>} \\
\f{<p>}\z{周易玩辭}\f{(}\z{十六}\f{<003>\smlb}\z{四冊}\f{)}\z{ }\f{\smlb(}\z{項世安}\f{)}\z{ 吳草}\f{<007>}\z{易纂言}\f{(}\z{六}\f{<003>\smlb}\z{四冊}\f{)}\z{ }\f{(}\z{吳澄}\f{)</p>} \\
\f{<p>}\z{周易說}\f{<008>(}\z{四}\f{<003>\smlb}\z{四冊}\f{)}\z{ }\f{(}\z{羅倫}\f{)}\z{ 周易象指}\f{??}\z{決録}\f{(}\z{七}\f{<003>\smlb}\z{六冊}\f{)}\z{ }\f{(}\z{態過}\f{)</p>} \\
\f{<pb b>} \\
\f{<p>}\z{周易筆意}\f{(}\z{十五}\f{<003>\smlb}\z{三冊}\f{)}\z{ }\f{\smlb(}\z{陶廷奎}\f{)}\z{ 學易記}\f{(}\z{五卷}\f{\smlb}\z{三冊}\f{)}\z{ }\f{(}\z{金賁亨}\f{)</p>} \\
\f{<p>}\z{易象大指}\f{(}\z{八卷}\f{\smlb}\z{四冊}\f{)}\z{ }\f{(}\z{薛甲}\f{)}\z{ 易象鮮}\f{(}\z{六}\f{<003>\smlb}\z{二冊}\f{)}\z{ }\f{(}\z{劉濂}\f{)</p>} \\
\f{<p>}\z{周易億}\f{(}\z{四}\f{<003>\smlb}\z{王文定公遺書}\f{)}\z{ }\f{(}\z{王道}\f{)}\z{ }\f{(}\z{載}\f{)}\z{ 周易傳義補疑}\f{(}\z{十二}\f{<003>\smlb}\z{六冊}\f{)}\z{ }\f{(}\z{姜寶}\f{)</p>} \\
\f{<p>}\z{周易生生篇}\f{(}\z{三}\f{<003>\smlb}\z{三冊}\f{)}\z{ }\f{(<006>}\z{濬}\f{)}\z{ 周易義訓}\f{(}\z{十卷}\f{\smlb}\z{十冊}\f{)}\z{ }\f{(}\z{任惟賢}\f{)</p>} \\
\f{<p>}\z{九正易因}\f{(}\z{十}\f{<003>\smlb}\z{四冊}\f{)}\z{ }\f{(}\z{李贄}\f{)}\z{ 易經象義}\f{(}\z{十}\f{<003>\smlb}\z{四冊}\f{)}\z{ }\f{(}\z{章潢}\f{)</p>} \\
\f{<p>}\z{易大象測}\f{(}\z{一}\f{<003>\smlb}\z{一冊}\f{)}\z{ }\f{(}\z{俱萬尚}\f{\smlb}\z{烈}\f{)}\z{ 易替測}\f{(}\z{二}\f{<003>\smlb}\z{二冊}\f{)</p>} \\
\f{<p>}\z{易大象說}\f{(}\z{二}\f{<003>\smlb}\z{二冊}\f{)}\z{ 周易就正略義}\f{(}\z{五}\f{<003>\smlb}\z{陳嘉謨}\f{)}\z{ }\f{(}\z{二冊}\f{)</p>} \\
\f{<p>}\z{周易指要}\f{(}\z{三}\f{<003>\smlb}\z{一冊}\f{)}\z{ }\f{(}\z{方社昌}\f{)}\z{ 沈氏易學}\f{(}\z{十二}\f{<003>)}\z{ }\f{(}\z{六冊}\f{)}\z{ }\f{(}\z{沈一貫}\f{)</p>} \\
\f{<p>}\z{郭司馬易鮮}\f{(}\z{十五}\f{<003>\smlb}\z{冊}\f{)}\z{ }\f{(}\z{六}\f{\smlb}\z{郭子章}\f{)}\z{ 乾坤二卦集鮮}\f{(}\z{三卷}\f{\smlb}\z{一冊}\f{)</p>} \\
\f{<pb c>} \\
\f{<p>}\z{易經繹}\f{(}\z{五}\f{<003>\smlb}\z{二冊}\f{)}\z{ }\f{(}\z{鄧元錫}\f{)}\z{ 易測}\f{(}\z{十}\f{<003>\smlb}\z{四冊}\f{)}\z{ }\f{(}\z{魯朝節}\f{)</p>} \\
\f{<p>}\z{易說}\f{(}\z{九}\f{<003>\smlb}\z{六冊}\f{)}\z{ }\f{(}\z{徐即登}\f{)</p>} \\
\f{<p>}\z{洗心齊讀易述}\f{(}\z{十七}\f{<003>)}\z{ }\f{(}\z{十三冊}\f{)}\z{ }\f{(}\z{潘士藻}\f{)</p>} \\
\f{<p>}\z{易經疑問}\f{(}\z{十二卷}\f{\smlb}\z{冊}\f{)}\z{ }\f{(}\z{八}\f{\smlb}\z{姚舜牧}\f{)}\z{ 易筌}\f{(}\z{六}\f{<003>\smlb}\z{六冊}\f{)}\z{ }\f{(}\z{焦竑}\f{)</p>} \\
\f{<p>}\z{像象管見}\f{(}\z{六卷}\f{\smlb}\z{四冊}\f{)}\z{ }\f{(}\z{錢一本}\f{)}\z{ 易知齊易說}\f{(}\z{一}\f{<003>\smlb}\z{一冊}\f{)}\z{ }\f{(}\z{李登}\f{)</p>} \\
\f{<p>}\z{易像管}\f{<009>(}\z{十五卷}\f{\smlb}\z{冊}\f{)}\z{ }\f{(}\z{八}\f{\smlb}\z{黃正憲}\f{)}\z{ 周易宗義}\f{(}\z{十二}\f{<003>\smlb}\z{十二冊}\f{)}\z{ }\f{(}\z{程汝}\f{\smlb}\z{繼}\f{)</p>} \\
\f{<p>}\z{周易繹}\f{<008>(}\z{七卷}\f{)}\z{ }\f{(}\z{四冊}\f{\smlb}\z{又更定易繹}\f{<008>)}\z{ }\f{(}\z{俱吳烔}\f{\smlb}\z{卷冊同}\f{)</p>} \\
\f{<p>}\z{周易畧義}\f{(}\z{乾坤至}\f{<010>}\z{八卦}\f{)}\z{ }\f{(}\z{一}\f{<003>)}\z{ }\f{(}\z{一冊}\f{)}\z{ }\f{(}\z{劉一焜}\f{)</p>} \\
\f{<p>}\z{易}\f{<012>(}\z{五}\f{<003>\smlb}\z{五冊}\f{)}\z{ }\f{(}\z{喻安性}\f{)}\z{ 易會}\f{(}\z{八卷}\f{\smlb}\z{四冊}\f{)}\z{ }\f{(}\z{鄒德溥}\f{)</p>} \\
\f{<pb d>} \\
\f{<p>}\z{易筏}\f{(}\z{六卷}\f{\smlb}\z{六冊}\f{)}\z{ }\f{(}\z{張燧輯}\f{)}\z{ 易}\f{<013>(}\z{三}\f{<003>\smlb}\z{一冊}\f{)}\z{ }\f{(}\z{陳履祥}\f{)</p>} \\
\f{<p><014>}\z{窩易因指}\f{(}\z{八卷}\f{\smlb}\z{八冊}\f{)}\z{ }\f{(}\z{張汝}\f{\smlb}\z{霖}\f{)}\z{ 易略}\f{(}\z{三卷}\f{\smlb}\z{一冊}\f{)}\z{ }\f{(}\z{陸夢龍}\f{)</p>} \\
\f{<p>}\z{鄭孔肩易臆}\f{(}\z{三}\f{<003>\smlb}\z{一冊}\f{)}\z{ }\f{(}\z{鄭圭}\f{)}\z{ 讀書心印}\f{(}\z{七}\f{<003>\smlb}\z{五冊}\f{)}\z{ }\f{(}\z{朱篁}\f{)</p>} \\
\f{<p>}\z{居易齊讀易雜言}\f{(}\z{一卷}\f{\smlb}\z{冊}\f{)}\z{ }\f{(}\z{一}\f{\smlb}\z{朱篁}\f{)</p>} \\
\f{<p>}\z{易賸}\f{(}\z{六}\f{<003>)}\z{ }\f{(}\z{三冊}\f{)}\z{ }\f{(}\z{王應遴}\f{)</p>} \\
\f{<p>}\z{易鮮俚語}\f{(}\z{一卷}\f{)}\z{ }\f{(}\z{林子分內集}\f{)}\z{ }\f{(}\z{林兆恩}\f{\smlb}\z{巳上俱詳說}\f{)</p>} \\
\f{<p>}\z{卦名鮮}\f{(}\z{一卷}\f{)}\z{ }\f{(}\z{王安石}\f{\smlb}\z{載本集}\f{)}\z{ 易象論}\f{)}\z{一卷}\f{)}\z{ }\f{)}\z{王安石}\f{\smlb}\z{載本集}\f{)</p>} \\
\f{<p>}\z{易}\f{<015>}\z{子問}\f{(}\z{三}\f{<003>)}\z{ }\f{(}\z{歐陽修}\f{\smlb}\z{載本集}\f{)}\z{ 張橫渠易說}\f{(}\z{三}\f{\smlb<003>)}\z{ }\f{(}\z{張載}\f{\smlb}\z{載本集}\f{)</p>} \\
\f{<p>}\z{李泰伯易論}\f{(}\z{一}\f{\smlb<003>)}\z{ }\f{(}\z{載本集}\f{)}\z{ 讀易私言}\f{(}\z{一卷}\f{)}\z{ }\f{(}\z{許衡}\f{\smlb}\z{載本集}\f{)</p>} \\
\f{</toc>} \\
\end{typeChineseSmall}

\end{example}

\vspace{-3mm}
\begin{note}
This example does not contain §<list>§ because page 0580 is part of a list that begins before this page and ends after this page.
\end{note}


\end{document}
