%!TEX TS-program = xelatex 
%!TEX encoding = UTF-8 Unicode 

\documentclass[fontsize=11pt, paper=a4, 
DIV15,
normalheadings,
parskip=half-, 
pointlessnumbers]{scrartcl}

\usepackage[british]{babel} 

\usepackage{fontspec,xltxtra,xunicode} 
\defaultfontfeatures{Mapping=tex-text} 

\setromanfont[Mapping=tex-text]{DejaVu Serif}
\setsansfont[Scale=MatchLowercase,Mapping=tex-text]{Helvetica} 
\setmonofont[Scale=1.0]{Courier New} 

\frenchspacing

\usepackage{graphicx}
\graphicspath{{../../bilder/}}

\usepackage{longtable}

\usepackage{philokalia}

\usepackage{yfonts}

%%%

\input{../../abbreviations/abbreviations_2}

%\usepackage{blacklettert1}

\begin{document}

\begin{center}
{\fontspec{Helvetica}{\LARGE \textbf{
Special Instructions for Cardano 1559
}}} \\[5mm]
\large Wolfgang Schmidle, Klaus Thoden, Malcolm D. Hyman

\normalsize Max Planck Institute for the History of Science, Berlin, Germany

\today
\end{center}

%\tableofcontents

\section{Cardano 1559}

In headings, the comma and the hyphen look like / (see e.g. p.160). In normal text the comma looks like /, but the hyphen looks like a normal \fraktur{-}. Rule: Type / as §/§, i.e. type what you see. Example. Normalisation (if / means hyphen, no space before or after, if / means comma, space before but not after) in post-processing? Any rule about this for Formax?

Contains \textswab{\LARGE  *u} and \textswab{\LARGE  "u}. Type both versions as §ü§, as in example 3 in the "Fraktur examples" section? Alternative: Special Instruction “Type \textswab{\LARGE  *u} as §{ue}§, etc.”, which makes sense only if there are both version in the text. Which is rare, I hope.

Contains  \textswab{\LARGE  *o}, but not \textswab{\LARGE  "o} ? Contains \textswab{\LARGE  *a} (e.g. p.160), but not \textswab{\LARGE  "a} ?  

Small u with o above: Type “u with o above” as §{uo}§.
\includegraphics[height=6mm]{uo}

Tilde to indicate a missing n or m, just as is Latin texts.

\includegraphics[height=6mm]{Cardano_der}: Type as §{der}§. Alternative: Do not mention it at all, make them type it as an unknown character.

Alchemy symbols (p.746): Type §<al>§ for each symbol. Do not add them to the list of unknown characters.

The drop caps are virtually undeciferable. Special instruction: Mark drop caps by <dc>. Do not transcribe the letter.

Roman numbers in the “Register” at the beginning of the text: small letters, especially the x looks strange; example.

Roman numbers as page numbers: the x looks different!


\end{document}
