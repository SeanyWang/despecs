%!TEX TS-program = xelatex 
%!TEX encoding = UTF-8 Unicode 

\documentclass[fontsize=11pt, paper=a4, 
DIV15,
normalheadings,
parskip=half-, 
pointlessnumbers]{scrartcl}

\usepackage[british]{babel} 

\usepackage{fontspec,xltxtra,xunicode} 
\defaultfontfeatures{Mapping=tex-text} 

\setromanfont[Mapping=tex-text]{DejaVu Serif}
\setsansfont[Scale=MatchLowercase,Mapping=tex-text]{Helvetica} 
\setmonofont[Scale=1.0]{Courier New} 

\frenchspacing

\usepackage{graphicx}
\graphicspath{{../../bilder/}}
\graphicspath{{./bilder/}}

\usepackage{longtable}

\usepackage{philokalia}

%%%

%!TEX TS-program = xelatex 
%!TEX encoding = UTF-8 Unicode 
%!TEX root = ../DESpecs.tex

\usepackage{xspace} \xspaceaddexceptions{”}
\usepackage{ifthen}

\newcommand{\ch}[1]		{chapter~\ref{#1}}
\newcommand{\sect}[1]		{section~\ref{#1}}
\newcommand{\fig}[1]		{figure~\vref{#1}}
\newcommand{\Fig}[1]		{Figure~\vref{#1}}
\newcommand{\tbl}[1]		{table~\vref{#1}}

\newcommand{\ie}			{i.e. }
\newcommand{\eg}			{e.g. }

\newcommand{\qq}			{\qquad}

\newcommand{\spitz}[1]		{\ensuremath{\langle}#1\ensuremath{\rangle}}

\newcommand{\mehrzeilen}[1][1]{\enlargethispage{#1\baselineskip}}


\newcommand{\unit}[1]		{\textsl{#1}}


% Zeilenabstand

\usepackage{setspace} 
\newenvironment{enum}{\begin{enumerate} \singlespacing} {\end{enumerate}}
\newenvironment{items}{\begin{itemize} \singlespacing} {\end{itemize}}


% verbatim

\usepackage{alltt}
\newcommand{\klein}{\small}
\newenvironment{exakt}[1][\small]{\singlespacing#1\begin{alltt}}{\end{alltt}}

\usepackage{shortvrb}
\MakeShortVerb{\§}


%%%%%%%%%%%%%%%%

\newenvironment{mainrule}{\textit{Rule}}{}

\newenvironment{example}{\textit{Example}}{}
\newenvironment{type}{\begin{alltt}}{\end{alltt}}

%\newfontfamily{\greek}[Scale=1]{Minion Pro} 
%\newfontfamily{\greek}[Scale=0.8]{Andale Mono} 
\newfontfamily{\greek}[Scale=0.8]{Courier New} 
\newenvironment{typeGreek}{\begin{alltt}\greek}{\end{alltt}}

\newenvironment{exception}{\textit{Exception}}{}

\newenvironment{clarification}{\textit{Clarification}}{}


\begin{document}

\begin{center}
{\fontspec{Helvetica}{\LARGE \textbf{
Special Instructions for three volumes of “Tartaglia”
\\[3mm]
(Addendum to Data Entry Specs 2.1.1) 
}}} \\[5mm]
\large Klaus Thoden

\normalsize Max Planck Institute for the History of Science, Berlin, Germany

\today
\end{center}

% \section{Introduction}
% These Special Instruction are for the three volumes
% §Tartaglia_1556_H5BAMGAN§, §Tartaglia_1556_MRV5C34S§ and
% §Tartaglia_1560_Q9B15RYG§. They deal with small pieces of text that
% interrupt the main text.

\section{Text Flows}


\begin{mainrule}
  These three books contain small passages of separate text inside
  paragraphs (see image). Please mark them as separate text flows with
  the tags §<tf>§ and §</tf>§. Type the <tf> and </tf> tags on separate
  lines.
\end{mainrule}

\begin{clarification}
Type the §<tf>§ and §</tf>§ tags on separate lines. On each page, type the first text flow before the second text flow.
\end{clarification}

\vspace{3mm}
\begin{sampleImage}[1: \, example of inset text flow. Please replace ”§<unknown1>§“ by the code you assign to this unknown character.]{Tartaglia_1560_Q9B15RYG-045.jpg}
\begin{typeLatin}
\someText \\
\bold{<p>} \someText
POngo anchora che $ia vna pezza di \someText\\
\someText\\
rai per la longhezza braccia 512.&per\bold{<lb/>}\\
\bold{<tf>}\\
Vna pezza di terra\bold{<lb/>}\\
lon. zu. 42. braccia 8. la proua è braccia 1.\bold{<lb/>}\\
te. zu. 23. braccia 3. la proua è braccia 6.\bold{<lb/>}\\
fa perti. 41. ta. 8. p.o. \bold{<unknown1>} o. la proua \bold{<unknown1>} 6.\bold{<lb/>}\\
\bold{</tf>}\\
la larghezza braccia 279. multiplicaraili detti brac- \bold{<lb/>}\\
\someText
\bold{</p>}
\end{typeLatin}
\end{sampleImage}


\end{document}

%%% Local Variables: 
%%% mode: latex
%%% TeX-master: t
%%% End: 
