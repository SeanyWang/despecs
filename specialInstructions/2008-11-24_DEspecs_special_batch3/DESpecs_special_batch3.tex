%!TEX TS-program = xelatex 
%!TEX encoding = UTF-8 Unicode 

\documentclass[fontsize=11pt, paper=a4, 
DIV15,
normalheadings,
parskip=half-, 
pointlessnumbers]{scrartcl}

\usepackage[british]{babel} 

\usepackage{fontspec,xltxtra,xunicode} 
\defaultfontfeatures{Mapping=tex-text} 

\setromanfont[Mapping=tex-text]{DejaVu Serif}
\setsansfont[Scale=MatchLowercase,Mapping=tex-text]{Helvetica} 
\setmonofont[Scale=1.0]{Courier New} 

\frenchspacing

\usepackage{graphicx}
\graphicspath{{./Bilder/}}

\usepackage{longtable}

\usepackage{philokalia}

%%%

%!TEX TS-program = xelatex 
%!TEX encoding = UTF-8 Unicode 
%!TEX root = ../DESpecs.tex

\usepackage{xspace} \xspaceaddexceptions{”}
\usepackage{ifthen}

\newcommand{\ch}[1]		{chapter~\ref{#1}}
\newcommand{\sect}[1]		{section~\ref{#1}}
\newcommand{\fig}[1]		{figure~\vref{#1}}
\newcommand{\Fig}[1]		{Figure~\vref{#1}}
\newcommand{\tbl}[1]		{table~\vref{#1}}

\newcommand{\ie}			{i.e. }
\newcommand{\eg}			{e.g. }

\newcommand{\qq}			{\qquad}

\newcommand{\spitz}[1]		{\ensuremath{\langle}#1\ensuremath{\rangle}}

\newcommand{\mehrzeilen}[1][1]{\enlargethispage{#1\baselineskip}}


\newcommand{\unit}[1]		{\textsl{#1}}


% Zeilenabstand

\usepackage{setspace} 
\newenvironment{enum}{\begin{enumerate} \singlespacing} {\end{enumerate}}
\newenvironment{items}{\begin{itemize} \singlespacing} {\end{itemize}}


% verbatim

\usepackage{alltt}
\newcommand{\klein}{\small}
\newenvironment{exakt}[1][\small]{\singlespacing#1\begin{alltt}}{\end{alltt}}

\usepackage{shortvrb}
\MakeShortVerb{\§}


%%%%%%%%%%%%%%%%

\newenvironment{mainrule}{\textit{Rule}}{}

\newenvironment{example}{\textit{Example}}{}
\newenvironment{type}{\begin{alltt}}{\end{alltt}}

%\newfontfamily{\greek}[Scale=1]{Minion Pro} 
%\newfontfamily{\greek}[Scale=0.8]{Andale Mono} 
\newfontfamily{\greek}[Scale=0.8]{Courier New} 
\newenvironment{typeGreek}{\begin{alltt}\greek}{\end{alltt}}

\newenvironment{exception}{\textit{Exception}}{}

\newenvironment{clarification}{\textit{Clarification}}{}


\begin{document}

\begin{center}
{\fontspec{Helvetica}{\LARGE \textbf{
Special Instructions for Barrow 1674 and Clavius 1586
\\[3mm]
(Addendum to Data Entry Specs 1.1.2) 
}}} \\[5mm]
\large Wolfgang Schmidle, Klaus Thoden, Malcolm D. Hyman

\normalsize Max Planck Institute for the History of Science, Berlin, Germany

\today
\end{center}

\section{Barrow 1674}

\begin{mainrule}
Rule 1: If a formula contains curly braces followed by text that is broken into separate lines, do not type a return after the lines. Instead, type §\\§ to separate the lines.
\end{mainrule}

%\vspace{3mm}
\begin{example}[1: \, Barrow p. 115]

\vspace{-5mm}
\begin{typeLatin}
fiat \{ I - R. I:: B C. B Z; \& \bold{\bs\bs} I. R:: D Z. D Y.
\end{typeLatin}

\end{example}

%\vspace{3mm}
\begin{example}[2: \, Barrow p. 120]

\vspace{-5mm}
\begin{typeLatin}
In vitro, $i C B = 15, erit \{ Z C = 9 \bold{\bs\bs} Z B = 24 \} & \{ A Z = 16. \bold{\bs\bs} A B = 40.
\end{typeLatin}

\end{example}

\begin{mainrule}
Rule 2: Type \, \includegraphics[height=5mm]{Barrow_kuerzer} \, as §--]§ and \, \includegraphics[height=5mm]{Barrow_laenger} \, as §[--§.
\end{mainrule}


\section{Clavius 1586}

\begin{mainrule}
Rule 1. In the case of a table element that spans more than one cell, repeat the symbol §#§ before the table element for each cell spanned by the element, e.g. §####§ for an element spanning four cells. 
\end{mainrule}

\begin{clarification}
Do not type spaces between the §#§ symbols in this case. This rule applies even if the spanning element is the first element in a table row.
\end{clarification}

\begin{note}
In all other cases there should be a space before and after each §#§ symbol.
\end{note}

\vspace{3mm}
\begin{mainrule}
Rule 2. Within tables cells, if text is broken into separate lines, do not type a return after the lines. Instead, type §\\§ followed by a space to separate the lines.
\end{mainrule}

%\vspace{3mm}
\begin{example}[: \, Clavius p. 11]

\vspace{-5mm}
\begin{typeLatin}
\bold{<tb>} \\
\someText \\
44. \bold{#} 20. \bold{#} Quod $secundo loco \bold{#} Quod primo loco \\
47. \bold{###} Secunda figura cum prima paginæ 48. locum permutet. \\
48. \bold{###} Prima figura \someText in tertio ca$u de- \bs\bs mon$trationis \someText paginæ 48. \\
\someText \\
\bold{</tb>} \\
\end{typeLatin}
%48. \bold{###} Prima figure cum $ecunda paginæ 47. locum permutet: & in tertio ca$u de- \bs\bs mon$trationis adhibeatur $secunda figura paginæ 48. \\

\end{example}

\begin{mainrule}
Rule 3. In the tables on pp. 144 - 179 and pp. 214 - 303, do not type the vertical text.
\end{mainrule}

\end{document}
