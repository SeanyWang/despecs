%!TEX TS-program = xelatex 
%!TEX encoding = UTF-8 Unicode 

\documentclass[fontsize=11pt, paper=a4, 
DIV15,
normalheadings,
parskip=half-, 
pointlessnumbers]{scrartcl}

\usepackage[british]{babel} 

\usepackage{fontspec,xltxtra,xunicode} 
\defaultfontfeatures{Mapping=tex-text} 

\setromanfont[Mapping=tex-text]{DejaVu Serif}
\setsansfont[Scale=MatchLowercase,Mapping=tex-text]{Helvetica} 
\setmonofont[Scale=1.0]{Courier New} 

\frenchspacing

\usepackage{graphicx}
\graphicspath{{./Bilder/}}

\usepackage{longtable}

\usepackage{philokalia}

%%%

\input{abbreviations/abbreviations_test}

\begin{document}

\begin{center}
{\fontspec{Helvetica}{\LARGE \textbf{
%Data Entry Specs 1.2 \\[3mm]
%\Large Special Instructions for Chinese Texts
Special Instructions for Chinese Texts in Batch 1 \\[3mm]
\Large (Addendum to the Data Entry Specs 1.2, version for Chinese texts)
}}} \\[5mm]
\large Wolfgang Schmidle, Martina Siebert, Martin Hofmann, \\[1mm] Klaus Thoden, Malcolm D. Hyman

\normalsize Max Planck Institute for the History of Science, Berlin, Germany

\today
\end{center}


\tableofcontents

\vspace{15mm}

\newcommand{\hash}{{\char"0023}}
\newcommand{\chin}[1]{{\fontspec{SimSun}{#1}}}
\newenvironment{typeChinese}{\begin{alltt}\s\begin{tabular}{@{}l}}{\end{tabular}\end{alltt}}



%In general: Give them the information about the heading levels in each book, or is the rule clear enough?

%\subsection{Text 11 (The Handwritten Text)}

%\subsubsection{Text Emendations}

%\begin{mainrule}
%If a character has been crossed out, mark it by §{ }§. If there is a corrected character next to the crossed-out character, mark this by §{ / }§.
%%If there is a line between two consecutive characters indicating that the order should be reversed, insert §~§ between the characters, but type them in the order as they appear in the text.
%If there is a line between two consecutive characters indicating that the order should be reversed, mark this by §{ ~ }§, but type the characters in the order as they appear in the text.
%\end{mainrule}

%
%\begin{tabular}{@{}ll}
%\parbox[b]{8cm}{
%\htsc{Examples} \\[25mm]
%\begin{typeChinese}
%xx\bold{\{}x\bold{\}}xx \\ \\
%xxxx\{x/x\}\{x/x\}xx \\ \\
%xx\bold{\tld}xxx \\ 
%\end{typeChinese}
%} & \qquad
%\includegraphics[height=6cm]{Specs-Bild2-kt}
%\end{tabular}

%Or something like §<del>x</del>§, §<repl>xx</repl>§ and §<rev>xx</re>§?


%\subsection{Text 14}
%Rule for mathematics, e.g. $a^2 + b^2 = c^2$ on p.57?

\section{Xifa shenji \, \chin{西法神機}}

\begin{mainrule}
Type the handwritten underlinings. Use §{ }§ for dotted lines. Use §[ ]§ for circles or circled lines. Use §( )§ for small circles. 
\end{mainrule}

\begin{clarification}
Do not treat circles as punctuation marks.
Do not mark the red colour.
\end{clarification}

%\begin{clarification}
%A circled line may consist of dots, circles and wedges.
%\end{clarification}

%\begin{tabular}{@{}ll}
%\parbox[b]{8cm}{
%\htsc{Example } \\[32mm]
%\begin{typeChinese}
%\bold{<cl>}\chin{其節短矣}\bold{</cl>}\chin{既猛烈而益} \\
%\chin{由是而}\bold{<cl>}\chin{別}\bold{</cl>}\chin{。使}\bold{<cl>}\chin{臨敵者氣}\bold{</cl>} \\
%\bold{<cl>}\chin{必固矣然必車製合宜}\bold{</cl>} \\
%\end{typeChinese}
%} & \qquad
%\includegraphics[height=6cm]{Specs-Bild3}
%\end{tabular}

\begin{tabular}{@{}ll}
\parbox[b]{125mm}{
\htsc{Example } \\[17mm]
\begin{typeChinese}
\bold{\{}\chin{其節短}\bold{\}[}\chin{矣}\bold{]}\chin{既猛烈而益} \\
\chin{由是而}\bold{[(}\chin{別}\bold{)]}\chin{使}\bold{[}\chin{臨敵者氣}\bold{]} \\
\bold{\{}\chin{必固}\bold{\}[}\chin{矣}\bold{]\{}\chin{然必車製合}\bold{\}[}\chin{宜}\bold{]} \\[12mm]
\end{typeChinese}
} & \qquad
\includegraphics[height=6cm]{Specs-Bild3}
\end{tabular}

\newpage

\section{Minchan lu yi \, \chin{閩產錄異}}

%\subsection{Repaired Text}

\begin{mainrule}
Repaired text is marked by §<rep>§ and §</rep>§.
\end{mainrule}

\begin{tabular}{@{}ll}
\parbox[b]{123mm}{
\htsc{Example} \\[105mm]
} & 
\includegraphics[height=12cm]{text17-1p15}
\end{tabular}

\begin{typeChinese}
\bold{<p>}\someText \\
\chin{天下春社後春分榖米上}\bold{<rep>}\chin{錦墩秋分秋社前斗米一}\bold{</rep>} \\
\chin{斗錢秋分秋社後斗米一斗豆}\bold{<rep><sm>}\chin{福州以早稻藳作薦}\bold{\bs\bs}\chin{呼草薦用晚稻病癩}\bold{</sm></rep></p>} \\
\bold{<h 2>}\chin{六十二黃}\bold{</h>}\chin{ }\bold{<p>}\chin{早稻也福州諺曰肩頭}\bold{<rep>}\chin{擔秧六十日言分}\bold{</rep>} \\
\chin{秧至收成僅六十餘日也蓋六十二}\bold{<rep>}\chin{黃之種成熟最}\bold{<rep>} \\
\chin{早各郡兩}\bold{<001>}\chin{者亦種之}\bold{</p>} \\
\end{typeChinese}

\begin{crossref}
The tag §<001>§ denotes a character that is not included in Unicode 5.1.0, see section 2.5.2 of the Data Entry Specs 1.2, version for Chinese texts.
The indentation of the paragraph is not marked because the paragraph is preceded by a sub-heading in the same line, see section 2.2.2 of the Data Entry Specs 1.2, version for Chinese texts.
\end{crossref}


\end{document}
