%!TEX TS-program = xelatex 
%!TEX encoding = UTF-8 Unicode 

\documentclass[fontsize=11pt, paper=a4, 
DIV15,
headings=normal, % normalheadings,
parskip=half-,
numbers=noenddot % pointlessnumbers
]{scrartcl}

\usepackage[british]{babel} 

\usepackage{fontspec,xltxtra,xunicode} 
\defaultfontfeatures{Mapping=tex-text} 

\setromanfont[Mapping=tex-text]{DejaVu Serif}
\setsansfont[Scale=MatchLowercase,Mapping=tex-text]{Helvetica} 
\setmonofont[Scale=1.0]{Courier New} 

\frenchspacing

\usepackage{graphicx}
\graphicspath{{../../Bilder/}}

\usepackage{longtable}

\usepackage{philokalia}

%%%

\input{../../abbreviations/abbreviations_1}

\begin{document}

\begin{center}
{\fontspec{Helvetica}{\LARGE \textbf{
Special Instructions for the “Vitruvius” Work Order
\\[3mm]
(Addendum to Data Entry Specs 2.0) 
}}} \\[5mm]
\large Wolfgang Schmidle, Klaus Thoden, Malcolm D. Hyman

\normalsize Max Planck Institute for the History of Science, Berlin, Germany

\today
\end{center}


\section{Handwritten Notes}

\begin{mainrule}
Handwritten underlining is marked by §<ul hd> </ul>§. Handwritten strokes through words are marked by §<st hd> </st>§. 
\end{mainrule}

%\begin{clarification}
%§<hd ul>§ and §<hd st>§ are typed on the same line as the underlined or struck-through words. However, all other handwritten notes are still marked by §<hd>§ on a separate line before or after the line of the main text the note is closest to. 
%% Make sure that each handwritten note is marked. 
%Do not type the note itself. 
%\end{clarification}

%\begin{clarification}
%(Alternative: All other handwritten notes are marked by §<hd>§. 
%If a handwritten note clearly belongs to a word in the text, type it there. 
%In particular, type §<hd ul>§ and §<hd st>§ on the same line as the underlined or struck-through words. 
%Otherwise, type §<hd>§ on a separate line after the line of the main text the note is closest to. 
%Do not type the note itself.)
%\end{clarification}

%\begin{clarification}
%All other handwritten notes are marked by §<hd>§. Place the §<hd>§ tag
%%where it best describes 
%according to the position of the handwritten note. If the handwritten note is in the margin, type §<hd>§ on a separate line after the line of the main text the note is closest to. 
%\end{clarification}

\begin{clarification}
All other handwritten notes or figures are marked by §<hd>§ on a separate line after the line of the main text to which the note or figure is closest. Do not type the note itself. Do not indicate the ink colour.
\end{clarification}

\vspace{5mm}
\begin{sampleImageSmall}[1: \, underlining]{height=8mm}{vitruv1511_123}

\vspace{-3mm}
\begin{typeLatin}
vita, \bold{<ul hd>}Epicurus\bold{</ul>} vero \\
\end{typeLatin}
\end{sampleImageSmall}

\begin{sampleImageSmall}[2: \, stroke through a word]{height=8mm}{vitruv1513_102}

\vspace{-3mm}
\begin{typeLatin}
uocabula, \bold{<st hd>}Pycno$tylos\bold{</st>}, id e$t \\
\end{typeLatin}
\end{sampleImageSmall}

%\newpage
\begin{sampleImage}[3: \, handwritten notes between lines and in the margin]{vitruv1513_99_neu2}
%\end{sampleImage}
%\begin{sampleImage}[3: \, handwritten notes between lines and in the margin]{vitruv1513_99_neu}

\vspace{-3mm}
\begin{typeLatin}
01  \bold{<p it>} ... \\
02  hypethros. Horum exprimuntur formationes, his \\
03  rationibus. In antis erit ædes, cum habebit in fron- \\
04  \bold{<hd><hd>} \\
05  te \bold{<st hd>}antas\bold{</st>} parietum, qui \bold{<st hd>}cellam\bold{</st>} circ\bs~ucludunt, & inter \\
06  \bold{<hd>} \\
07  antas in medio col\bs~unas duas $upra\bs'q; fa$tigium $ym- \\
08  metria ea collocatum, quæ in hoc libro fuerit per$cri- \\
09  \bold{<hd>} \\
10  pta. Huius autem exemplar erit ad tres fortunas, ex \\
11  \bold{<hd>} \\
12  tribus, quod e$t proxime portam collinam.\bold{<hd>} \\
13  \bold{</p>} \\
\end{typeLatin}
\end{sampleImage}

\begin{note}
The two §<hd>§ in line 04 denote the two crosses above the struck-through words. Line 06 denotes the handwritten note in the margin. Lines 09 and 11 denote handwritten notes between lines of printed text.
\end{note}

%\begin{sampleImageSmall}[4: \, a handwritten note in the margin]{height=8mm}{vitruv1513_102}

%\vspace{-3mm}
%\begin{typeLatin}
%bla \\
%\bold{<hd>} \\
%bla \\
%\end{typeLatin}
%\end{sampleImageSmall}

\vspace{5mm}
\begin{sampleImageSmall}[4: \, handwritten note within a figure]{width=8cm}{vitruv1513_81}

\begin{typeLatin}
\bold{<p it>} ... \\
telli, Iouis $tatoris, Hermodi, & ad Mariana honoris \\
& uirtutis $ine po$tico à Mutio facta.\bold{</p>} \\
\bold{<fig>} \\
\bold{<hd>} \\
\bold{</fig>} \\
\bold{<p it>}P$eudodipteros autem $ic collocatur, ut in fronte & \\
po$tico $int columnæ octonæ, in lateribus cum angula \\
... \bold{</p>} \\
\end{typeLatin}
\end{sampleImageSmall}


\section{Figure Descriptions in the Margin}

\begin{mainrule}
Marginal notes to the left or right of figures are figure descriptions. Mark them by §<desc> </desc>§.
\end{mainrule}

\begin{clarification}
Use one §<desc>§ tag for each paragraph. As always, type the §<desc>§ tags between the §<fig>§ and §</fig>§ lines. The position of the description in the left or right margin is not encoded.
\end{clarification}

\vspace{3mm}
\begin{sampleImageSmall}{width=12cm}{vitruv1511_81}

\begin{typeLatin}
\bold{<p>} ... \\
tres, e quibus vna plinthus cum cymatio fiat, altera echinus cum anulis, ter \\
tia hypotrachelio c\bs~otrahatur columnæ, ita vti in tertio libro de ionicis e$t \\
$criptum.\bold{</p>} \\
\bold{<fig>} \\
\bold{<desc>}a. cymati\bs~u\bold{</desc>} \\
\bold{<desc>}b. plinthus\bold{</desc>} \\
\bold{<desc>}c. echinus c\bs~u \\
anulis\bold{</desc>} \\
\bold{<desc>}d. pars \bs~q hy- \\
potrachilio \\
contrahitur \\
columnæ\bold{</desc>} \\
\bold{<var>}a b c d\bold{</var>} \\
\bold{</fig>} \\
\bold{<p>}Epi$tylii altitudo vnius moduli c\bs~u tenia & guttis, tenia moduli $eptima, \\
guttarum longitudo $ub tenia contra triglyphos alta c\bs~u regula parte $ex- \\
... \bold{</p>} \\
\end{typeLatin}
\end{sampleImageSmall}


\section{Latin Abbreviations and Diacritics}

\begin{mainrule}
In addition to the ligatures given in the Data Entry Specs 2.0, please resolve the abbreviations §{ur}§ and §{us}§.
\end{mainrule}

\begin{clarification}
As always, a tilde above a letter is marked marked by §\~§. Especially above the small letter i, make sure to distinguish between the normal dot above the i and a tilde.
\end{clarification}

\vspace{3mm}
\begin{liste}[ : \, additional abbreviations and diacritics]

\includegraphics[height=8mm]{vitruv1511_121} \quad
\includegraphics[height=8mm]{ornatus} \quad
\includegraphics[height=8mm]{intercurrunt} \quad

\vspace{-3mm}
\begin{typeLatin}
$truant\li{ur}   ornat\li{us}    \bs~itercurrunt
\end{typeLatin}
\end{liste}

%\begin{tabular}{lllllllllll} 
%\\
%\includegraphics[height=8mm]{neulig_longsi} & §nascit{ur}§ \\[3mm]
%\includegraphics[height=8mm]{neulig_longsi} & §nascit{ur}§ \\[3mm]
%\includegraphics[height=8mm]{neulig_longsi} & §nascit{ur}§ \\[3mm]
%\end{tabular}


\end{document}
