%!TEX TS-program = xelatex 
%!TEX encoding = UTF-8 Unicode 

\documentclass[fontsize=11pt, paper=a4, 
DIV15,
normalheadings,
parskip=half-, 
pointlessnumbers]{scrartcl}

\usepackage[british]{babel} 

\usepackage{fontspec,xltxtra,xunicode} 
\defaultfontfeatures{Mapping=tex-text} 

\setromanfont[Mapping=tex-text]{DejaVu Serif}
\setsansfont[Scale=MatchLowercase,Mapping=tex-text]{Helvetica} 
\setmonofont[Scale=1.0]{Courier New} 

\frenchspacing

\usepackage{graphicx}
\graphicspath{{./Bilder/}}

\usepackage{longtable}

\usepackage{philokalia}

%%%

%!TEX TS-program = xelatex
%!TEX encoding = UTF-8 Unicode

\usepackage{xspace} \xspaceaddexceptions{”}
\usepackage{ifthen}

\newcommand{\ch}[1]		{chapter~\ref{#1}}
\newcommand{\sect}[1]		{section~\ref{#1}}
\newcommand{\fig}[1]		{figure~\vref{#1}}
\newcommand{\Fig}[1]		{Figure~\vref{#1}}
\newcommand{\tbl}[1]		{table~\vref{#1}}

\newcommand{\ie}			{i.e. }
\newcommand{\eg}			{e.g. }

\newcommand{\qq}			{\qquad}

\newcommand{\spitz}[1]		{\ensuremath{\langle}#1\ensuremath{\rangle}}

\newcommand{\mehrzeilen}[1][1]{\enlargethispage{#1\baselineskip}}


% Zeilenabstand

\usepackage{setspace}
\newenvironment{enum}{\begin{enumerate} \singlespacing} {\end{enumerate}}
\newenvironment{items}{\begin{itemize} \singlespacing} {\end{itemize}}


% verbatim

\usepackage{verbatim}

\usepackage{alltt}
\newcommand{\klein}{\small}
\newenvironment{exakt}[1][\small]{\singlespacing#1\begin{alltt}}{\end{alltt}}

\usepackage{shortvrb}
\MakeShortVerb{\§}


%%%%%%%%%%%%%%%%%%%%%%%%%%%%%%%%

% xml commands
% use for any xml markup, brackets supplied
\newcommand{\xml}[1]{§<#1>§}
% use for milestone tags
\newcommand{\xms}[1]{§<#1/>§}
% closing element markup
\newcommand{\xmcl}[1]{§</#1>§}
% full xml markup example
\newcommand{\xmex}[1]{§#1§}
% attribute markup, the first argument is the element name
\newcommand{\attr}[2]{§@#2§}

\newcommand{\bold}{\textbf}
% ligature
\newcommand{\li}[1]{\bold{\{}#1\bold{\}}}

\newcommand{\xs}{\scriptsize}
\newcommand{\s}{\footnotesize}

%

\newcommand{\bs}{\textbackslash}
\newcommand{\tld}{\textasciitilde}

\newcommand{\tocspace}{\addtocontents{toc}{\protect\vspace{1mm}}}

\newcommand{\unicode}[1]{{\fontspec{Apple Symbols}{\Large #1}}}
\newcommand{\§}{{\char"00A7}}

%%%%%%%%%%%%%%%%

\newcommand{\htsc}[1]{\emph{#1}}
\newcommand{\lig}[1]{\fontspec{Hoefler Text}{\Large #1}}
\newcommand{\fraktur}[1]{{\fontspec{BreitkopfFraktur}{\LARGE #1}}}

%

\newenvironment{mainrule}{}{}
\newenvironment{mainruleLessImportant}{}{}
\newenvironment{clarification}{\s}{}
\newenvironment{exception}{\htsc{Exception:}}{}
\newenvironment{note}{\textbf{Please note:}}{}
\newenvironment{crossref}{\s\ensuremath{\longrightarrow}}{}

%

\newenvironment{sampleImage}[2][]{\parbox{\linewidth}{{\htsc{Example#1}} \\[3mm] \includegraphics[width=\linewidth]{#2}}}{}
\newenvironment{sampleImageSmall}[3][]{\parbox{\linewidth}{{\htsc{Example#1}} \\[3mm] \includegraphics[#2]{#3}}}{}

\newenvironment{example}[1][]{\htsc{Example#1} \\}{}
\newenvironment{exampleTest}[2][]{\parbox{\linewidth}{\htsc{Example #1} \\[3mm] #2}}{} % ??

\newenvironment{liste}[1][]{\htsc{List#1} \\}{}
\newenvironment{tabelle}[1][]{\htsc{Table#1} \\}{}

%

\newenvironment{typeLatin}{\begin{alltt}\s\begin{tabular}{@{}l}}{\end{tabular}\end{alltt}}

\newfontfamily{\greek}[Scale=0.95]{Courier New}
\newenvironment{typeGreek}{\begin{alltt}\greek\s\begin{tabular}{@{}l}}{\end{tabular}\end{alltt}}

\newenvironment{typeMath}{\begin{alltt}\begin{tabular}{l}}{\end{tabular}\end{alltt}}

%

\newfontfamily{\muh}[Scale=0.9]{DejaVu Serif}
\newcommand{\someText}{...} % {{\muh\textit{(some text)}}}
\newcommand{\untranscribedText}{...} % {{\muh\textit{(some untranscribed text)}}}
\newcommand{\notTranscribed}{{\muh\textit{(not transcribed)}}}
\newcommand{\missingText}[1]{{\muh\textit{(#1)}}}

%% Chinese bits
\newenvironment{typeChinese}{\begin{alltt}\s\begin{tabular}{@{}l}}{\end{tabular}\end{alltt}}

\newcommand{\chin}[1]{{\fontspec{Sun-ExtA}{#1}}}
\newcommand{\sunExtA}[1]{{\fontspec{Sun-ExtA}{#1}}}
\newcommand{\sunExtB}[1]{{\fontspec{Sun-ExtB}{#1}}}

\newcommand{\mincho}[1]{{\fontspec{MS Mincho}{#1}}}
\newcommand{\hira}[1]{{\fontspec{HiraMinPro-W3}{#1}}}

\newcommand{\f}[1]{\bold{#1}} % f für fett
\newcommand{\z}[1]{\chin{#1}} % z für Zeichen


\begin{document}

\begin{center}
{\fontspec{Helvetica}{\LARGE \textbf{
Special Instructions for Tables
\\[3mm]
(Addendum to Data Entry Specs 1.1.2) 
}}} \\[5mm]
\large Wolfgang Schmidle, Klaus Thoden, Malcolm D. Hyman

\normalsize Max Planck Institute for the History of Science, Berlin, Germany

\today
\end{center}

\section{Ghetaldi 1603}

\begin{mainrule}
Rule 1. In the case of a table element that spans more than one cell, repeat the symbol §#§ before the table element for each cell spanned by the element, e.g. §####§ for an element spanning four cells. 
\end{mainrule}

\begin{clarification}
Do not type spaces between the §#§ symbols in this case. This rule applies even if the spanning element is the first element in a table row.
\end{clarification}

\begin{note}
In all other cases there should be a space before and after each §#§ symbol.
\end{note}

\vspace{3mm}
\begin{mainrule}
Rule 2. Within tables cells, if text is broken into separate lines, do not type a return after the lines. Instead, type §\\§ followed by a space to separate the lines.
\end{mainrule}

\vspace{3mm}
\begin{example}[: \, Ghetaldi p. 79]

% \vspace{-5mm}
\begin{typeLatin}
\bold{<tb>} \\
\bold{#######} Tabula ad inueniendam qualitatem \bold{\bs\bs} Auri, ex grauitate quam ha- \bold{\bs\bs} bet in aere & aqua. \\
Qualitas Auri. \bold{#} Grauitas Auri in aere. \bold{####} Grauitas Auri in Aqua. \bold{#} Mi$t\~u ex Arg. \bold{\bs\bs} & ære. \\
Part. \bold{#} Lib. \bold{#} Vnc. \bold{#} Scrup. \bold{#} Gran. \bold{#} Num. Fract. \bold{#} Part. \\
24 \bold{#} 1 \bold{#} 11. \bold{#} 8. \bold{#} 20. \bold{#} 372 \bold{#} 0 \\
23 \bold{#} 1 \bold{#} 11. \bold{#} 8. \bold{#} 5. \bold{#} 765 \bold{#} 1 \\
\\\someText \\ \\
1 \bold{#} 1 \bold{#} 10. \bold{#} 18. \bold{#} 16. \bold{#} 576 \bold{#} 23 \\
0 \bold{#} 1 \bold{#} 10. \bold{#} 18. \bold{#} 1. \bold{#} 969 \bold{#} 24 \\
Part. \bold{#} Lib. \bold{###} Communis Denomin. fract. \bold{#} 1767 \bold{#} Part. \\
\bold{</tb>} \\
\bold{<tb>} \\
\bold{###} Tabella Partis pro \bold{\bs\bs} portionalis Deno- \bold{\bs\bs} minatorum Auri. \\
Pars proportio \bold{\bs\bs} nalis Auri in \bold{\bs\bs} partibus. 24. \bold{##} Differ\~etia Gra \bold{\bs\bs} uitatum Auri \bold{\bs\bs} in aqua. \\
Part. \bold{#} Gran. \bold{#} Num: Fract. \\
1 \bold{#} 0. \bold{#} 1088 \\
2 \bold{#} 1. \bold{#} 409 \\
\\\someText \\ \\
23 \bold{#} 14. \bold{#} 286 \\
24 \bold{#} 14. \bold{#} 1374 \\
Part. \bold{##} Denom. Fract. com. \bold{\bs\bs} 1767 \\
\bold{</tb>} \\
\end{typeLatin}

\end{example}

\section{Stevin 1605}

\begin{mainrule}
Type structures as the one on p. 9/10 as a table with three columns.
\end{mainrule}

\section{Euclid 1607}

All three examples are taken from p.506. Examples 1 and 3 cannot be easily represented as tables and should be typed as normal text. Example 2 should be typed as a table.

\begin{example}[1]

\begin{typeLatin}
\bold{<p>}9 8 \\
18 17 16 \\
27 26 25 24 \\
36 35 34 33 32 \\
45 44 43 42 41 40 \\
54 53 52 51 50 49 48 \\
63 62 61 60 59 58 57 56\bold{</p>} \\
\end{typeLatin}

\end{example}

\begin{example}[2]

\begin{typeLatin}
\bold{<tb>} \\
Ordo propor. \bold{\bs\bs} $uperquadr. \bold{#} 1 \bold{#} 2 \bold{#} 3 \bold{#} 4 \bold{#} 5 \bold{#} 6 \bold{#} 7 \bold{#} 8 \bold{#} &c. \\
Multit. med. \bold{\bs\bs} numerorum.. \bold{#} 3 \bold{#} 7 \bold{#} 11 \bold{#} 15 \bold{#} 19 \bold{#} 23 \bold{#} 27 \bold{#} 31 \bold{#} &c. \\
\bold{</tb>} \\
\end{typeLatin}

\end{example}

\begin{example}[3]

\begin{typeLatin}
\bold{<p>}11. 10. 9. 8. 7.  \\
22. 21. 20. 19. 18. 17. 16. 15. 14.  \\
33. 32. 31. 30. 29. 28. 27. 26. 25. 24. 23. 22. 21.  \\
44. 43. 42. 41. 40. 39. 38. 37. 36. 35. 34. 33. 32. 31. 30. 29. 28\bold{</p>}  \\
\end{typeLatin}

\end{example}

%\begin{note}
%In the examples in these Special Instructions, the §<tb>§ and §</tb>§ are typed on separate lines. We will make this a general rule in a future version of the Data Entry Specs.
%\end{note}

\end{document}
